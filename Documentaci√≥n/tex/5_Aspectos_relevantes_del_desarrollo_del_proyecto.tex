\capitulo{5}{Aspectos relevantes del desarrollo del proyecto}

\section{Etapas del proyecto}\label{etapas-del-proyecto}

\subsection{Investigación}
Durante la primera etapa del proyecto, se investigó acerca de \textit{FIWARE} y sus distintas aplicaciones, y se decidió enfocar la aplicación a usar el modelo información \textit{NGSI-LD} \ref{modelo-ngsi-ld}. 
Esta decisión fue impulsada por el reciente crecimiento de la plataforma \textit{FIWARE} y el modelo \textit{NGSI-LD}, gracias a la colaboración de las empresas más significativas en el desarrollo \textit{IOT}, como Telefónica \footnote{Proyecto de Telefónica de \textit{fiware-orion}, el cual lleva 6 años en marcha y a día de hoy se sigue ampliando. \url{https://github.com/telefonicaid/fiware-orion/tree/master}}.


\subsection{Despliegue del \textit{Orion Context Broker} y \textit{MongoDB}}

Tras un proceso de búsqueda e investigación de \textit{OpenData} reales que usaran el modelo \textit{NGSI-LD}, por ser este un estándar de uso libre, se descubrió que el número de conjuntos de datos que siguiesen este modelo de información era muy excaso aún. Por lo que se decidió crear un \textit{mock server} que usara NGSI-LD, en lugar de los distintos formatos que están en uso a día de hoy (\textit{JSON}, \textit{GEOJSON}, \textit{CSV}, \textit{XML}...), estandarizando así el uso de un mismo formato para todos los conjuntos de datos. Esto favorecería el desarrollo y mantenimiento de todas las aplicaciones que consuman este tipo de datos, no solo para las \textit{Smart Cities} sino en muchos otro ámbitos como \textit{Industria 4.0}.

Se comenzó con la instalación del \textit{Orion Context Broker} para la gestión de la información en tiempo real y, \textit{MongoDB}, que actúa como la base de datos subyacente para almacenar esta información de contexto. Todo ello se desplegó en un \textit{Docker}.
\imagen{OCB-Running}{\textit{Orion Context Broker} y la base de datos \textit{MongoDB} en funcionamiento}{.7}

Posteriormente se elaboró un \textit{script} que permitiese al usuario gestionarlo de manera más rápida, ya que, el \textit{Orion Context Broker} se gestiona mediante comandos, y para una mayor reusabilidad y mejor mantenimiento.
Con este mini programa se permite al usuario:
\begin{itemize}
    \item Iniciar y para el servidor, es decir encender el \textit{docker} de del \textit{Orion Context Broker} y el de la \textit{MongoDB}.
    \item Mostrar y borrar las entidades almacenadas.
    \item Iniciar y para el simulador. Explicaré su funcionamiento más adelante en \ref{simulador}.
    \item Instalar y desinstalar los \textit{dockers}.
\end{itemize}

\imagen{GestorOCB}{Programa gestor del OCB}{.8}

En esta etapa mediante la herramienta Postman\footnote{Herramienta para probar y gestionar APIs}, se hicieron pruebas con el Orion Context Broker, creando, consultando, actualizando y borrando entidades el servidor. Así se podía verificar que el servidor funcionaba correctamente y estaba preparado para almacenar las distintas entidades.
Las cuales posteriormente serían consultadas, actualizadas o eliminadas desde la aplicación con peticiones \textit{GET}\footnote{Se utiliza para solicitar datos de un recurso específico.}, \textit{PATCH}\footnote{Se utiliza para hacer actualizaciones parciales a un recurso existente.} o \textit{DELETE}\footnote{Se utiliza para eliminar un recurso específico.} de \textit{HTTP} y poder consultadas con peticiones \textit{GET} de \textit{HTTP} en el puerto 1026, que es el usado por el servidor.
\imagen{POST - Postman}{Ejemplo de uso de POST en Postman}{.9}\label{fig:post-postman}

\subsection{Simulador de conjunto de datos usando el modelo \textit{NGSI-LD}}\label{simulador}

Con las primeras cargas de datos se pudo observar que el modelo información \textit{NGSI-LD} todavía no está siendo muy usado por las diferentes \textit{APIs} de datos, por lo que se decidió crear un generador de datos que usara este modelo. Este es usado a modo de simulador, el cual genera de un conjunto de datos de \textit{parkings} de la ciudad de Burgos simulando recibir datos de sensores instalados en dichos \textit{parkings} y lo adapta al modelo \textit{NGSI-LD}. 

\imagen{GET - Postman}{Ejemplo de uso de GET en Postman}{.9}\label{fig:get-postman}


Con esto se pudo probar que el \textit{OCB}\footnotemark{} estaba funcionando correctamente y pudiendo así leer y escribir datos de manera correcta. También se introdujeron algunas entidades diferentes, no solo \textit{parkings}, para probar que los distintos conjuntos de datos podían convivir sin problema en el mismo\textit{OCB}\footnotemark[\value{footnote}]. 

\footnotetext{Orion Context Broker}

Para un mejor entendimiento de la estructura del proyecto, se adjunta una imagen con un esquema de la estructura \ref{fig:Estructura}.

\imagen{Estructura}{Estructura del proyecto}{.9}

\subsection{Desarrollo de la aplicación}

Como el objetivo del proyecto es realizar una aplicación funcional y, el número de conjuntos de datos disponibles con el modelo de información\textit{NGSI-LD} aún es mínimo, se comenzó con el desarrollo de la aplicación con los formatos tradicionales que se están usando a día de hoy (\textit{JSON}, \textit{GEOJSON}, \textit{CSV}, \textit{XML}...) y se dejó el \textit{OCB}\footnotemark[\value{footnote}] preparado para que cuando el número de conjuntos que utilicen el modelo de información\textit{NGSI-LD} sea mayor, se pueda utilizar.

Para el desarrollo de la aplicación se hizo un análisis de los posibles perfiles de usuarios que tendría y sus necesidades. Se clasificaron en dos grupos:

\begin{itemize}
    \item \textbf{Usuario estándar:} usaría la aplicación para acceder a un conjunto de datos en concreto en un momento puntual. Su principal objetivo sería obtener información acerca de un conjunto de manera rápida y precisa. Por ejemplo, un usuario quiere decidir a qué biblioteca de su ciudad acudir, por ello, consulta en conjunto y analiza las posibles opciones en función de los parámetros que más le interesen (aforo, horario, proximidad...).
    \item \textbf{Usuario avanzado:} este usuario sería un perfil más próximo a un analista de datos, que desearía consultar uno o varios conjuntos, durante un periodo de tiempo determinado con una frecuencia establecida.
\end{itemize}

Una vez establecidos los perfiles y las necesidades a cubrir, se empezó con el análisis y diseño de la aplicación. Y posteriormente su desarrollo, el cuál, ha sido complejo ya que requería de cubrir las necesidades de los usuarios actuales y de usuarios futuros con un modelo de información diferente, el modelo NGSI-LD, mencionado en el apartado \ref{modelo-ngsi-ld}.

Para su desarrollo se ha tenido en cuenta que entre los objetivos generales, se encuentren objetivos tales como permitir al usuario tener una experiencia sencilla, cómoda y rápida. Es por ello que un usuario puede marcar que parámetros desea visualizar de un conjunto, o marcar varios conjuntos como favoritos y poder acceder desde el menú de inicio de manera rápida.

Las funcionalidades principales de dicha aplicación, basándose en los perfiles de los usuarios potenciales y en los objetivos y necesidades a cubrir son:

\begin{itemize}
    \item \textbf{Consulta de conjuntos de datos en tiempo real:} esta sería la principal funcionalidad de la aplicación, permite al usuario consultar cualquier conjunto de datos de los que se encuentren incorporados en la aplicación. \imagen{puntos_de_carga_Valencia}{Ejemplo de consulta de datos en tiempo real}{.9} 
    \item \textbf{Añadir conjuntos de datos:} esta funcionalidad permite a los usuarios añadir nuevos conjuntos de datos, los cuales estarán visibles al resto de usuarios.
    \item \textbf{Añadir registros:} cualquier usuario puede comenzar a registrar un conjunto de datos con una frecuencia determinada, si la frecuencia es mayor con la que se actualizan en la \textit{API} de datos de donde se extraen, se notifica al usuario para que corrija su elección de periodicidad.
    \item \textbf{Gestionar registros:} un usuario que tenga uno más o registros guardándose para él, puede descargar o eliminar lo registros que desee.
    \item \textbf{Favoritos:} a la hora de consultar un conjunto de datos, un usuario puede marcar o desmarcar de favoritos un conjunto, esto modificará su ventana de \textit{home}, donde se muestran los conjuntos favoritos del usuario para poder acceder a ellos de una manera más rápida.
    \item \textbf{Personalizar vista de conjuntos de datos:} en los mapas, el usuario puede modificar los \textit{tooltips}, para poder mostrar los parámetros que él desee. También puede ordenar la tabla por orden alfabético según el campo que desee.
    \imagen{PCval2}{Ejemplo de consulta de datos en tiempo real 2}{.9}
\end{itemize}

Otras funcionalidades principales que solo tienen los administradores son:
\begin{itemize}
    \item \textbf{Gestionar conjuntos:} un administrador puede eliminar conjuntos. Esto permite que en caso de que un administrador detecte o se le comunique que alguno de los enlaces está roto o los datos están corruptos, podrá quitar ese conjunto de la base de datos.
    \item \textbf{Gestionar usuarios:} un administrador puede consultar todos los usuarios que se encuentran en la aplicación, cambiar el rol de algún usuario o eliminarlo.
    \item \textbf{Gestionar traducciones:} un puede añadir traducciones que se emplean para hacer que la lectura de datos por parte del usuario sea más cómoda y legible. Por ejemplo, un atributo que sea "PlLibr" en un conjunto de datos, se vea traducido como "Plazas libres"
\end{itemize}

\subsection{Revisión de código}
CUANDO SE TERMINE EL USO DE SONARQUBE

\begin{comment}

Este apartado pretende recoger los aspectos más interesantes del desarrollo del proyecto, comentados por los autores del mismo.
Debe incluir desde la exposición del ciclo de vida utilizado, hasta los detalles de mayor relevancia de las fases de análisis, diseño e implementación.
Se busca que no sea una mera operación de copiar y pegar diagramas y extractos del código fuente, sino que realmente se justifiquen los caminos de solución que se han tomado, especialmente aquellos que no sean triviales.
Puede ser el lugar más adecuado para documentar los aspectos más interesantes del diseño y de la implementación, con un mayor hincapié en aspectos tales como el tipo de arquitectura elegido, los índices de las tablas de la base de datos, normalización y desnormalización, distribución en ficheros3, reglas de negocio dentro de las bases de datos (EDVHV GH GDWRV DFWLYDV), aspectos de desarrollo relacionados con el WWW...
Este apartado, debe convertirse en el resumen de la experiencia práctica del proyecto, y por sí mismo justifica que la memoria se convierta en un documento útil, fuente de referencia para los autores, los tutores y futuros alumnos.
\end{comment}