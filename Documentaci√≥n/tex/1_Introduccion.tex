\capitulo{1}{Introducción}
\begin{comment}
Descripción del contenido del trabajo y del estructura de la memoria y del resto de materiales entregados.
\end{comment}

En la última década, la demanda de acceso instantáneo a datos en tiempo real ha crecido exponencialmente. Esta necesidad se debe en parte a la proliferación de dispositivos conectados y al avance de la tecnología de la información y la comunicación (TIC). Los seres humanos, tanto a nivel individual como colectivo, requieren información rápida y precisa para tomar decisiones informadas, mejorar la eficiencia operativa y optimizar la gestión de recursos \cite{khajeh2020, patel2016, gubbi2013}.

Para acceder a la información en tiempo real, es fundamental el uso de dispositivos equipados con sensores y conexión a Internet. Estos dispositivos, que pueden variar desde teléfonos inteligentes y relojes inteligentes hasta sensores ambientales y cámaras de seguridad, son capaces de recopilar, procesar y transmitir datos de manera continua. Este flujo constante de datos permite a los sistemas y aplicaciones reaccionar instantáneamente a las condiciones cambiantes y proporcionar información actualizada a los usuarios.

El concepto de Internet de las Cosas (\textit{IoT}, por sus siglas en inglés) se refiere a la interconexión de dispositivos y objetos cotidianos a través de Internet, permitiéndoles enviar y recibir datos. Según Gubbi et al. (2013), IoT es ''una visión en la que 'cosas' (objetos físicos) están integradas con electrónica, software, sensores y conectividad de red para recopilar e intercambiar datos'' \cite{gubbi2013}. Esto crea un ecosistema en el que los datos pueden ser utilizados para mejorar la eficiencia, productividad y calidad de vida.

En el contexto de las \textit{Smart Cities}, el \textit{IoT} juega un papel crucial. Los sensores y dispositivos IoT distribuidos por toda la ciudad pueden monitorizar el estado de diferentes infraestructuras, precios, disponibilidad, y otros parámetros esenciales en tiempo real.

Este trabajo propone una aplicación que facilita al usuario el acceso a todos estos datos, permitiendo así, a cualquier persona acceder a los datos de la ciudad que desee en tiempo real, de manera rápida y precisa. 

Existen otras aplicaciones que permiten el acceso a dichos datos, pero muchas de ellas suelen estar enfocadas en un conjunto de datos concreto, como pueden ser los \textit{parkings}, en una ciudad específica, o simplemente son \textit{APIs} de datos que ofrecen los datos en crudo mediante ficheros \textit{JSON}, \textit{CSV} u otros formatos determinados.
En esta aplicación, el usuario puede acceder al conjunto de datos que desee, de la ciudad que desee, y tener una visión gráfica y personalizada de los distintos conjuntos de datos. Además, esta aplicación tiene la ventaja de estar desarrollada de manera web, esto permite el acceso rápido a cualquier usuario y en cualquier dispositivo que tenga acceso a un navegador web, sin necesidad de descargarse ninguna aplicación.

\section{Estructura de la memoria}\label{estructura-de-la-memoria}

La memoria sigue la siguiente estructura:

\begin{itemize}
\tightlist
\item
  \textbf{Introducción:} breve introducción al problema a resolver y la
  solución propuesta en este trabajo. Estructura de la memoria y listado de materiales
  entregados.
\item
  \textbf{Objetivos del proyecto:} objetivos que
  persigue el desarrollo del proyecto. Tanto los objetivos marcados por los requisitos del software y los objetivos de carácter técnico.
\item
  \textbf{Conceptos teóricos:} explicación de los conceptos teóricos más importantes para la comprensión del proyecto.
\item
  \textbf{Técnicas y herramientas:} técnicas y herramientas utilizadas durante las diferentes etapas del proyecto y una explicación y/o comparación con otras que finalmente no se han utilizado.
\item
  \textbf{Aspectos relevantes del desarrollo:} aspectos más importantes del proyecto, como sus etapas, metodología, el proceso de aprendizaje...
\item
  \textbf{Trabajos relacionados:} 
\item
  \textbf{Conclusiones y líneas de trabajo futuras:} Conclusiones del proyecto, conclusiones técnicas, conclusiones personales y futuras mejoras.
\end{itemize}

La memoria contiene los siguientes anexos:

\begin{itemize}
\tightlist
\item
  \textbf{Plan del proyecto software:} 
\item
  \textbf{Especificación de requisitos del software:}
\item
  \textbf{Especificación de diseño:} 
\item
  \textbf{Manual del programador:} 
\item
  \textbf{Manual de usuario:} 
\end{itemize}





