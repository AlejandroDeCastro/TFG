\capitulo{2}{Objetivos del proyecto}
\begin{comment}
Este apartado explica de forma precisa y concisa cuales son los objetivos que se persiguen con la realización del proyecto. Se puede distinguir entre los objetivos marcados por los requisitos del software a construir y los objetivos de carácter técnico que plantea a la hora de llevar a la práctica el proyecto.
\end{comment}

El objetivo principal del proyecto es el desarrollo de una aplicación que permita consultar los datos abiertos de \textit{Smart Cities} utilizando la plataforma FIWARE \cite{fiware}. 

\section{Objetivos marcados por los requisitos del \textit{software}}\label{objetivos-generales}

\begin{itemize}
\tightlist
\item
  Permitir a cualquier usuario el acceso a los distintos conjuntos de datos, que los servicios de \textit{IT} hayan publicado en abierto, de manera visual, rápida y precisa.
\item
  Ofrecer al usuario una experiencia sencilla y cómoda, enfocada a cualquier tipo de público. También se busca que esta experiencia sea personalizada, pudiendo personalizar la visualización de cada conjunto por cada usuario.
\item
  Permitir al usuario guardar un registro de los conjuntos de datos que desee para un posterior análisis.
\item
  Facilitar al usuario la incorporación de nuevos conjuntos de datos, disponibles para todos los usuarios.
\end{itemize}


\section{Objetivos de carácter técnico}\label{objetivos-personales}

\begin{itemize}
\tightlist

\item Investigar sobre la implantación del protocolo de \textit{IoT} \textit{FIWARE} y generar una aplicación sencilla de publicación de datos y ponerlo en marcha.
\item Aprender sobre el desarrollo de aplicaciones web.
\item Ampliar y consolidar conocimientos sobre análisis y tratamiento de datos.
\item Usar \textit{GitHub} para el control de versiones.
\item Implementar una base de datos en la aplicación para guardar los diferentes datos necesarios.
\item Usar una metodología ágil \textit{Scrum} para el desarrollo del proyecto.
\item Aplicar y ampliar los conocimientos sobre \textit{Python}, desarrollo web, bases de datos, gestión de proyectos, estructuras de datos y análisis de \textit{software} adquiridos en el grado.
\item Aplicar y ampliar los conocimientos adquiridos durante el grado.
\item Ampliar los conocimientos de la herramienta de desarrollo de documentos \LaTeX{}.


\end{itemize}