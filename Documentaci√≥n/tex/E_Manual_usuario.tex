\apendice{Documentación de usuario}

\section{Introducción}
En este anexo


\section{Requisitos de usuarios}

Navegador

\section{Instalación}

La aplicación al ser desplegada carece de 
El usuario no tiene que descargar

\section{Manual del usuario}


\subsection{Iniciar sesión}\label{inicio-sesión}
Cuando el usuario entra en el enlace de la aplicación, lo primero que se le muestra es la pantalla de inicio \ref{fig:Pantalla - Login} de sesión donde debe introducir su usuario y contraseña.
\imagen{Pantalla - Login}{Pantalla de inicio de sesión}

En caso de introducir mal la contraseña se le notificará al usuario con un mensaje de error como se muestra en la imagen \ref{fig:Pantalla - Login - Contraseña incorrecta}.
\imagen{Pantalla - Login - Contraseña incorrecta}{Mensaje de contraseña incorrecta}

Y en caso de no existir un usuario con ese nombre en la base de datos se le notificará también con un error como el que se muestra en la imagen \ref{fig:Pantalla - Login - Usuario no encontrado}.
\imagen{Pantalla - Login - Usuario no encontrado}{Mensaje de usuario no encontrado}


\subsection{Crear una cuenta}\label{crear-cuenta}

Si el usuario no tiene una cuenta en la aplicación puede crear una pulsando en el enlace que dice "¿No tiene cuenta aún? Créala aquí", será redirigido a la pantalla  \ref{fig:Pantalla - Crear cuenta}, donde puede crear una cuenta introduciendo un nombre de usuario, nombre completo y la contraseña dos veces.
\imagen{Pantalla - Crear cuenta}{Pantalla de registro}

En caso de existir un usuario con ese nombre de usuario se le notificará con un mensaje como el de la imagen \ref{fig:Pantalla - Crear cuenta - Usuario existe}.
\imagen{Pantalla - Crear cuenta - Usuario existe}{Mensaje usuario ya registrado}

Y en el caso de no coincidir las dos contraseñas también se le notificará al usuario.
\imagen{Pantalla - Crear cuenta - Contraseñas no coinciden}{Mensaje contraseñas no coinciden}

Una vez introducido los datos correctamente, el usuario solo tiene que pulsar el botón "Registrar" y la cuenta será creada. Posteriormente será redirigido la pantalla de login \ref{fig:Pantalla - Login} donde deberá introducir sus datos para continuar.


\subsection{Pantalla principal}\label{home}
Una vez iniciada sesión en la aplicación el usuario será redirigido a la pantalla principal, la cual varía según el rol del usuario. Si se acaba de crear la cuenta por defecto será un usuario estándar y la pantalla que verá será la representada en la imagen \ref{fig:Pantalla - Home}:
\imagen{Pantalla - Home}{Pantalla principal para un usuario estándar}

Por el contrario, si es un usuario con permiso de administrador, esta será la pantalla principal para ese usuario:
\imagen{Pantalla - Home - Administrador}{Pantalla principal para un usuario administrador}

En esta pantalla principal ambos usuarios pueden ver en la parte inferior izquierda los conjuntos que tienen marcados en favoritos y acceder a ellos pulsando sobre ellos o seleccionando la ciudad y el conjunto en el cuadrado inferior de la parte derecha de la pantalla. Ambas opciones redirigirán al usuario a la pantalla de visualizar conjunto \ref{visualizar-conjunto}.

Los usuarios estándar podrán consultar dentro del recuadro verde que como se muestra en la imagen \ref{fig:Pantalla - Home - Registros} los registros que tienen en ese momento guardando.
\imagen{Pantalla - Home - Registros}{Cuadro de registros de la pantalla principal}

Los usuarios con rol de administrador tendrán en su defecto un menú con tres opciones que redirigirán al usuario a la interfaz de gestión de usuarios \ref{gestión-usuarios}, la lista de conjuntos disponibles \ref{consultar-conjuntos} o la pantalla de gestión de traducciones \ref{gestión-traducciones}, como se muestra en la imagen \ref{fig:Pantalla - Home - Administrador - Opciones}.
\imagen{Pantalla - Home - Administrador - Opciones}{Opciones del administrador en la pantalla principal}

Ambos tipos de usuarios, en la barra de navegación, podrán consultar conjuntos disponibles \ref{consultar-conjuntos} pulsando en el botón "Conjuntos", consultar los registros que tienen actualmente guardando, pulsando en el botón "Registros" \ref{consultar-registros}, consultar la pantalla de ayuda \ref{consulta-ayuda} con el botón "Ayuda" o cerrar la sesión pulsando el botón en rojo "Cerrar sesión". Si el usuario selecciona esta opción será redirigido a la pantalla de inicio de sesión \ref{inicio-sesión}

\subsection{Consultar los conjuntos de datos disponibles}\label{consultar-conjuntos}
Desde la pantalla \ref{fig:Pantalla - Conjuntos} el usuario puede consultar una lista con los conjuntos disponibles en la aplicación y acceder a ellos pulsando en el botón "consultar" azul de cada conjunto. Será direccionado al la visualización del conjunto seleccionado \ref{visualizar-conjunto}.
\imagen{Pantalla - Conjuntos}{Pantalla que mmuestra al usuario todos los conjuntos disponibles}

También el usuario puede ordenar de manera ascendente o descendente los conjuntos en función de la ciudad, el tipo de conjunto o el formato, pulsando en el nombre del campo en el cabecero de la tabla.

Si desea añadir un conjunto el usuario solo tiene que rellenar los campos de la parte superior de  la pantalla y posteriormente pulsar el botón "Añadir", y automáticamente se añadirá el conjunto a la lista.

Los administradores tienen un botón rojo "Eliminar" en cada conjunto. Esto permite a los administradores eliminar el conjunto que seleccionen, podemos observar la pantalla de los administradores en \ref{fig:Pantalla - Conjuntos - Administrador}.
\imagen{Pantalla - Conjuntos - Administrador}{Pantalla de conjuntos de un administrador}

\subsection{Consultar registros}\label{consultar-registros}
Desde la pantalla de visualizar registros \ref{fig:Pantalla - Registros}, se pueden ver todos los registros que ese usuario tiene en ese momento.
También, se pueden añadir nuevos registros seleccionando el conjunto, añadiendo una periodicidad y pulsando en el botón "Guardar". Automáticamente se comenzará con la captura de datos.
Por último también se pueden eliminar los registros que el usuario quiera dejar de capturar pulsando el botón "Eliminar"

\subsection{Visualizar conjunto en tiempo real}\label{visualizar-conjunto}



\subsection{Gestionar traducciones}\label{gestión-traducciones}
Un administrador desde la pantalla \ref{fig:Pantalla - Traducciones} puede consultar todas las traducciones que existen en la aplicación. Estas traducciones son aplicadas a los campos de los conjuntos de datos.
El administrador puede añadir nuevas traducciones rellenando los campos "Campo original" y "Traducción" y pulsando el botón "Añadir".
\imagen{Pantalla - Traducciones}{Pantalla de gestión de traducciones}

\subsection{Gestionar usuarios}\label{gestión-usuarios}
Desde la pantalla de gestión de usuarios \ref{fig:Pantalla - Traducciones} un administrador puede consultar los usuarios registrados en la aplicación, modificar su rol mediante el desplegable del campo "Rol" o eliminarlo desde el botón de "Eliminar".
\imagen{Pantalla - Usuarios}{Pantalla de gestión de usuarios}

\subsection{Ayuda}\label{consulta-ayuda}
Gracias a la pantalla de ayuda \ref{fig:Pantalla - Ayuda}, se pueden consultar las preguntas más frecuentes realizadas por otros usuarios y sus respuestas. También se puede acceder en el apartado de enlaces de interés a una guía elaborada con la herramienta \textit{GitBook}\footnote{\url{https://www.gitbook.com}}, al repositorio del código de la aplicación en \textit{GitBook}\footnote{\url{https://www.github.com}} o abrir un correo nuevo con un administrador como destinatario para contactar con él.
\imagen{Pantalla - Ayuda}{Pantalla de ayuda}

\subsection{Errores}\label{errores}
Si en algún punto de la aplicación algo no saliese como debería, se mostraría la pantalla de error \ref{fig:Pantalla - Error}, en la cuál se puede observar cual ha sido el error y reportarlo si se desea, también el usuario puede volver directamente la pantalla principal \ref{fig:Pantalla - Home}
\imagen{Pantalla - Error}{Pantalla de error}