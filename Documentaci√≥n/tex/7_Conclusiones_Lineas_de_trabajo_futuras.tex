\capitulo{7}{Conclusiones y Líneas de trabajo futuras}
\begin{comment}
Todo proyecto debe incluir las conclusiones que se derivan de su desarrollo. Éstas pueden ser de diferente índole, dependiendo de la tipología del proyecto, pero normalmente van a estar presentes un conjunto de conclusiones relacionadas con los resultados del proyecto y un conjunto de conclusiones técnicas. 
Además, resulta muy útil realizar un informe crítico indicando cómo se puede mejorar el proyecto, o cómo se puede continuar trabajando en la línea del proyecto realizado. 
\end{comment}

\section{Conclusiones}

Tras finalizar el proyecto se han obtenido diferentes conclusiones, las cuales he clasificado en tres grupos.
\begin{itemize}
    \item \textbf{Conclusiones del proyecto}
    \item \textbf{Conclusiones técnicas}
    \item \textbf{Conclusiones personales}
\end{itemize}

\subsection{Conclusiones del proyecto}
El proyecto ha concluido con éxito, cumpliendo todos y cada uno de los objetivos marcados al comienzo del mismo. Se ha conseguido realizar una aplicación funcional, rápida y accesible; cubriendo las necesidades de los usuarios de manera personalizada y un costo de recursos mínimo.

\subsection{Conclusiones técnicas}
\begin{itemize}
    \item Es escalable y fácilmente mantenible. Permite añadir nuevos formatos de manera rápida y simple.
    \item Tiene una interfaz amigable con el usuario, permitiendo que pueda personalizar los conjuntos que desea consultar y añadir nuevos. 
    \item La aplicación es robusta, tiene un tratamiento de excepciones preparado para cubrir todos los errores. 
    \item Posee cifrado para que las contraseñas estén siempre seguras.
    \item Su implementación y uso es simple.
\end{itemize}

\subsection{Conclusiones personales}
Uno de los objetivos de este proyecto es aplicar y ampliar los conocimientos adquiridos durante el grado, con ello, he podido desarrollar nuevas habilidades de gestión de proyectos, análisis y diseño de sistemas, tratamientos de datos en tiempo real, administración de bases de datos, programación orientada a objetos, estructuras de datos, interacción hombre-maquina, entre otros.

Gracias a este proyecto he podido mejorar mucho la capacidad de adaptación a nuevos retos de diversa índole. También he podido trabajar la capacidad de resolución de problemas, ya que, como se ha detallado con anterioridad, en este proyecto, se ha llegado a puntos en los que los obstáculos encontrados eran muy superiores a la capacidad de recursos y tiempo que se tenía. Aún así, se ha sabido sacar el proyecto adelante con gran éxito.


\section{Posibles mejoras futuras}

Como se ha detallado con anterioridad la aplicación posee acceso a multitud de conjuntos de datos, pero todos ellos con en formatos estándares actuales. En este proyecto se ha ido más allá, y se ha buscado poder tener una solución futura al nuevo \textit{IOT}\footnote{Internet Of Things}, con el nuevo modelo de información NGSI-LD, el cual ya ha sido modelado e integrado, como se puede ver en el simulador \ref{simulador}.

Cygnus

Internacionalizar