\capitulo{3}{Conceptos teóricos}

Para entender correctamente el proyecto, procedo a explicar los conceptos fundamentales que servirán de cimientos para el desarrollo del proyecto.

En primer lugar se explica la plataforma FIWARE. Posteriormente el modelo de información NGSI-LD que servirá de estándar y el Orion Context Broker, el encargado de almacenar y gestionar toda la información.

\section{Introducción a \textit{IoT}}

El Internet de las Cosas (\textit{IoT}, por sus siglas en inglés) es un concepto que se refiere a la interconexión de dispositivos y objetos físicos a través de internet. Estos dispositivos, equipados con sensores, software y otras tecnologías, pueden recopilar y compartir datos con otros dispositivos y sistemas a través de redes. \textit{IoT} permite la automatización y el control remoto de procesos y sistemas, lo que resulta en una mayor eficiencia, ahorro de costos y mejora en la calidad de vida \cite{McEwen2013, Ashton2009, Gartner2023}.

\subsection{Componentes Clave de \textit{IoT}}

\begin{itemize}
    \item \textbf{Dispositivos y Sensores}: Recogen datos del entorno o del sistema en el que están integrados.
    \item \textbf{Conectividad}: Permite la transmisión de datos a través de redes (Wi-Fi, Bluetooth, 5G, etc.).
    \item \textbf{Plataformas de Datos}: Almacenan y procesan los datos recogidos por los dispositivos.
    \item \textbf{Interfaz de Usuario}: Permite a los usuarios interactuar con el sistema \textit{IoT}, controlando dispositivos y visualizando datos.
\end{itemize}

\section{\textit{Fiware}}
\textit{Fiware} es una iniciativa que ofrece un conjunto de estándares abiertos para el manejo y compartición de datos, impulsada por la Unión Europea para el desarrollo y despliegue global de aplicaciones \textit{IoT}. Proporciona \textit{APIs} abiertas y componentes para gestionar información de contexto en soluciones inteligentes, gemelos digitales y espacios de datos \cite{fiware}. En resumen, \textit{Fiware} es una base tecnológica que permite crear soluciones innovadoras en áreas como las Ciudades Inteligentes, Agricultura Inteligente, Energía Inteligente, Industria Inteligente y Gestión del Agua. 

\section{\textit{NGSI-LD} y su aplicación en Ciudades Inteligentes}\label{modelo-ngsi-ld}

\textit{NGSI-LD} es un modelo de información y \textit{API} para editar, consultar y suscribirse a información de contexto. Está destinado a facilitar el intercambio abierto y la compartición de información estructurada entre diferentes partes interesadas \cite{etsi_ngsi_ld}. Se utiliza en diversos ámbitos de aplicación, como Ciudades Inteligentes, Industria Inteligente, Agricultura Inteligente, y más generalmente para el Internet de las Cosas, Sistemas Ciberfísicos y Gemelos Digitales \cite{etsi_iot}.

El modelo \textit{NGSI-LD} se compone de entidades, atributos y metadata.

Las entidades de definen como objetos \textit{JSON} y representan cada uno de los objetos físicos o lógicos, como sensores o estancias. En ese proyecto serían \textit{parkings}, bibliotecas, puntos de carga, etc).
Todas las entidades deben poseer:
\begin{itemize}
    \item \textbf{identificador} para poder relacionar cada entidad con la información que proporciona
    \item \textbf{tipo}, el cual indicará la naturaleza del objeto, si es una \textit{parking}, un tipo de sensor, etc.
\end{itemize}

Las entidades poseen también otras propiedades denominadas atributos, el segundo componente de este modelo.
Los atributos aportan información sobre la entidad y tienen la siguiente estructura:
\begin{itemize}
    \item \textbf{Nombre:} define el atributo. Por ejemplo, "plazas libres".
    \item \textbf{Tipo:} representa el tipo de atributo, no corresponde con los tipos de JSON ya que \textit{NGSI} tiene sus propios valores. Por ejemplo, "\textit{Number}".
    \item \textbf{Valor:} contiene el valor del atributo. Opcionalmente aquí puede haber metadatos, como la unidad o marca de tiempo.
\end{itemize}

Los metadatos tienen la misma estructura que los atributos: nombre, tipo y valor. Estos nos proporcionan valores adicionales sobre el atributo, como puede ser las unidades del valor.

\subsection{Ejemplo modelo \textit{NGSI-LD}}\label{ejemplo-uso-ngsi}
Imagina que estamos desarrollando una aplicación para monitorizar el tráfico en una ciudad inteligente. Queremos obtener información sobre el estado actual de los semáforos en una intersección específica.

\begin{enumerate}
    \item \textbf{Entidad (Entity)}: Representa un semáforo en una intersección. Cada semáforo tiene un ID único y se encuentra en una ubicación específica.
    \item \textbf{Atributos (Attributes)}: Los atributos de un semáforo podrían incluir:
        \begin{itemize}
            \item \textbf{Estado (State)}: 
                \begin{itemize}
                    \item \textbf{type}: \texttt{Property}
                    \item \textbf{value}: Puede ser “rojo”, “verde” o “amarillo”.
                \end{itemize}
            \item \textbf{Tiempo restante (Remaining Time)}: 
                \begin{itemize}
                    \item \textbf{type}: \texttt{Property}
                    \item \textbf{value}: El tiempo restante en segundos para cambiar de estado.
                    \item \textbf{unitCode}: \texttt{SEC}
                \end{itemize}
            \item \textbf{Ubicación (Location)}: 
                \begin{itemize}
                    \item \textbf{type}: \texttt{GeoProperty}
                    \item \textbf{value}: Las coordenadas geográficas del semáforo (latitud, longitud).
                \end{itemize}
        \end{itemize}
    \item \textbf{Contexto (Context)}: En nuestro modelo, el contexto sería la información sobre los semáforos en la ciudad. Por ejemplo:
        \begin{itemize}
            \item \textbf{Semáforo 1}:
                \begin{itemize}
                    \item \textbf{ID}: \texttt{TrafficLight:sem1}
                    \item \textbf{type}: \texttt{Semáforo}
                    \item \textbf{Estado}: 
                        \begin{itemize}
                            \item \textbf{type}: \texttt{Property}
                            \item \textbf{value}: “rojo”
                        \end{itemize}
                    \item \textbf{Tiempo restante}: 
                        \begin{itemize}
                            \item \textbf{type}: \texttt{Property}
                            \item \textbf{value}: 30
                            \item \textbf{unitCode}: \texttt{SEC}
                        \end{itemize}
                    \item \textbf{Ubicación}: 
                        \begin{itemize}
                            \item \textbf{type}: \texttt{GeoProperty}
                            \item \textbf{value}: (40.7128, -74.0060)
                        \end{itemize}
                \end{itemize}
            \item \textbf{Semáforo 2}:
                \begin{itemize}
                    \item \textbf{ID}: \texttt{TrafficLight:sem2}
                    \item \textbf{type}: \texttt{Semáforo}
                    \item \textbf{Estado}: 
                        \begin{itemize}
                            \item \textbf{type}: \texttt{Property}
                            \item \textbf{value}: “verde”
                        \end{itemize}
                    \item \textbf{Tiempo restante}: 
                        \begin{itemize}
                            \item \textbf{type}: \texttt{Property}
                            \item \textbf{value}: 15
                            \item \textbf{unitCode}: \texttt{SEC}
                        \end{itemize}
                    \item \textbf{Ubicación}: 
                        \begin{itemize}
                            \item \textbf{type}: \texttt{GeoProperty}
                            \item \textbf{value}: (40.7127, -74.0059)
                        \end{itemize}
                \end{itemize}
        \end{itemize}
    \item \textbf{Consulta (Query)}: Podemos consultar el estado actual de un semáforo específico utilizando su ID. Por ejemplo:
        \begin{itemize}
            \item “¿Cuál es el estado actual del semáforo con ID ‘TrafficLight:sem1’?”
        \end{itemize}
    \item \textbf{Respuesta (Response)}: La aplicación recibiría la información del semáforo solicitado y podría mostrarla al usuario.
\end{enumerate}

En resumen, \textit{NGSI-LD} nos permite modelar y acceder a información de contexto de manera estructurada, lo que es fundamental para construir aplicaciones inteligentes y conectadas \cite{fiware_ngsi_ld}. Las entidades y atributos definidos en \textit{NGSI-LD} se utilizan para representar datos en aplicaciones \textit{Fiware}.


\section{\textit{Orion Context Broker}}\label{orion-context-broker}
El componente principal y esencial de cualquier plataforma o solución desarrollada con \textit{Fiware} es el \textit{Orion Context Broker (OCB)}, que desempeña una función crucial en cualquier solución inteligente: gestionar, consultar y actualizar la información de contexto.

El OCB permite que las entidades, conocidas como proveedores de contexto (como los sensores), publiquen información de contexto para que esté disponible para otras entidades, conocidas como consumidores de contexto, que están interesadas en procesar dicha información. Por ejemplo, una aplicación que desea consultar el estado de los semáforos para calcular la ruta más corta en un momento dado es un consumidor de contexto. Los proveedores y consumidores de contexto pueden ser cualquier aplicación o incluso otros componentes dentro de la plataforma \textit{Fiware} \cite{FIWARE-Orion}.

El \textit{OCB} es un servidor que implementa una \textit{API} basada en el modelo de información \textit{NGSI}, permitiendo varias operaciones:

\begin{itemize}
    \item \textbf{Registrar aplicaciones de proveedores de contexto}: por ejemplo, un sensor de temperatura en una habitación.
    \item \textbf{Actualizar información de contexto}: por ejemplo, enviar actualizaciones sobre la temperatura.
    \item \textbf{Ser notificado de cambios en la información de contexto}: por ejemplo, cuando la temperatura cambia, o a intervalos regulares, como obtener la temperatura cada minuto.
    \item \textbf{Consultar información de contexto}: Orion almacena la información de contexto actualizada desde las aplicaciones, lo que permite que las consultas se resuelvan basándose en esta información.
\end{itemize}

El servidor \textit{OCB} siempre está escuchando en un puerto que generalmente es el 1026. el \textit{OCB} utiliza la base de datos es \textit{MongoDB} para almacenar el estatus actual de las entidades, no se almacena información histórica de sus cambios. Para este propósito se debe utilizar una base de datos externa al \textit{OCB} \cite{orion}, como es el caso de \textit{Cygnus}\footnote{\textit{Cygnus} es un componente de la plataforma \textit{Fiware} que actúa como un conector de datos y tiene la función principal de persistir la información de contexto gestionada por el \textit{Orion Context Broker} en diferentes sistemas de almacenamiento a largo plazo. \textit{Cygnus} recibe notificaciones de \textit{Orion} cuando hay actualizaciones en la información de contexto y las almacena en bases de datos u otros sistemas de almacenamiento seleccionados por el usuario \cite{fiware_cygnus}.} , componente el cual se estudió su implementación en este proyecto como se detalla en el anexo A, apartado de \textit{Planificación temporal} \textit{Sprint 5: Simulador (14/02/2024 - 14/03/2024)}.

En la imagen \ref{fig:OCBserver} extraída de la documentación oficial\footnote{Imagen extraída de \url{https://fiware-training.readthedocs.io/es-mx/latest/ecosistemaFIWARE/ocb/}} se puede apreciar el esquema de funcionamiento del \textit{OCB}:
\imagen{OCBserver}{Diagrama de \textit{OCB}}{.6}


\begin{comment}
En aquellos proyectos que necesiten para su comprensión y desarrollo de unos conceptos teóricos de una determinada materia o de un determinado dominio de conocimiento, debe existir un apartado que sintetice dichos conceptos.

Algunos conceptos teóricos de \LaTeX{} \footnote{Créditos a los proyectos de Álvaro López Cantero: Configurador de Presupuestos y Roberto Izquierdo Amo: PLQuiz}.

%\section{Secciones}

Las secciones se incluyen con el comando section.

\subsection{Subsecciones}

Además de secciones tenemos subsecciones.

\subsubsection{Subsubsecciones}

Y subsecciones. 


\section{Referencias}

Las referencias se incluyen en el texto usando cite~\cite{wiki:latex}. Para citar webs, artículos o libros~\cite{koza92}, si se desean citar más de uno en el mismo lugar~\cite{bortolot2005, koza92}.


\section{Imágenes}

Se pueden incluir imágenes con los comandos standard de \LaTeX, pero esta plantilla dispone de comandos propios como por ejemplo el siguiente:

\imagen{escudoInfor}{Autómata para una expresión vacía}{.5}



\section{Listas de items}

Existen tres posibilidades:

\begin{itemize}
	\item primer item.
	\item segundo item.
\end{itemize}

\begin{enumerate}
	\item primer item.
	\item segundo item.
\end{enumerate}

\begin{description}
	\item[Primer item] más información sobre el primer item.
	\item[Segundo item] más información sobre el segundo item.
\end{description}
	
\begin{itemize}
\item 
\end{itemize}

\section{Tablas}

Igualmente se pueden usar los comandos específicos de \LaTeX o bien usar alguno de los comandos de la plantilla.

\tablaSmall{Herramientas y tecnologías utilizadas en cada parte del proyecto}{l c c c c}{herramientasportipodeuso}
{ \multicolumn{1}{l}{Herramientas} & App AngularJS & API REST & BD & Memoria \\}{ 
HTML5 & X & & &\\
CSS3 & X & & &\\
BOOTSTRAP & X & & &\\
JavaScript & X & & &\\
AngularJS & X & & &\\
Bower & X & & &\\
PHP & & X & &\\
Karma + Jasmine & X & & &\\
Slim framework & & X & &\\
Idiorm & & X & &\\
Composer & & X & &\\
JSON & X & X & &\\
PhpStorm & X & X & &\\
MySQL & & & X &\\
PhpMyAdmin & & & X &\\
Git + BitBucket & X & X & X & X\\
Mik\TeX{} & & & & X\\
\TeX{}Maker & & & & X\\
Astah & & & & X\\
Balsamiq Mockups & X & & &\\
VersionOne & X & X & X & X\\
} 
\end{comment}