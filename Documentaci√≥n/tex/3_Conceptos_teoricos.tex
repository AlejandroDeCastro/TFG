\capitulo{3}{Conceptos teóricos}

Para entender correctamente el proyecto, procedo a explicar los conceptos fundamentales que servirán de cimientos para el desarrollo del proyecto.

En primer lugar se explica la plataforma FIWARE. Posteriormente el modelo de información NGSI-LD que servirá de estándar y el Orion Context Broker, el encargado de almacenar y gestionar toda la información.

\section{FIWARE}
FIWARE es una plataforma impulsada por la Unión Europea para el desarrollo y despliegue 
global de aplicaciones de Internet del Futuro. Proporciona APIs abiertas y componentes para 
gestionar información de contexto en soluciones inteligentes, gemelos digitales y espacios de 
datos \cite{fiware}. En resumen, FIWARE es una base tecnológica que permite crear soluciones 
innovadoras en áreas como \textit{Smart Cities}, Agricultura Inteligente, Energía Inteligente, Industria Inteligente y Gestión del Agua. 

\section{NGSI-LD y su Aplicación en Ciudades Inteligentes}\label{modelo-ngsi-ld}

NGSI-LD es un modelo de información y API para editar, consultar y suscribirse a información de contexto. Está destinado a facilitar el intercambio abierto y la compartición de información estructurada entre diferentes partes interesadas \cite{etsi_ngsi_ld}. Se utiliza en diversos ámbitos de aplicación, como Ciudades Inteligentes, Industria Inteligente, Agricultura Inteligente, y más generalmente para el Internet de las Cosas, Sistemas Ciberfísicos y Gemelos Digitales \cite{etsi_iot}.

\subsection{Ejemplo modelo NGSI-LD}\label{ejemplo-uso-ngsi}
Imagina que estamos desarrollando una aplicación para monitorizar el tráfico en una ciudad inteligente. Queremos obtener información sobre el estado actual de los semáforos en una intersección específica.

\begin{enumerate}
    \item \textbf{Entidad (Entity)}: Representa un semáforo en una intersección. Cada semáforo tiene un ID único y se encuentra en una ubicación específica.
    \item \textbf{Atributos (Attributes)}: Los atributos de un semáforo podrían incluir:
        \begin{itemize}
            \item \textbf{Estado (State)}: Puede ser “rojo”, “verde” o “amarillo”.
            \item \textbf{Tiempo restante (Remaining Time)}: El tiempo restante en segundos para cambiar de estado.
            \item \textbf{Ubicación (Location)}: Las coordenadas geográficas del semáforo.
        \end{itemize}
    \item \textbf{Contexto (Context)}: En nuestro modelo, el contexto sería la información sobre los semáforos en la ciudad. Por ejemplo:
        \begin{itemize}
            \item \textbf{Semáforo 1}:
                \begin{itemize}
                    \item \textbf{ID}: “sem1”
                    \item \textbf{Estado}: “rojo”
                    \item \textbf{Tiempo restante}: 30 segundos
                    \item \textbf{Ubicación}: (latitud, longitud)
                \end{itemize}
            \item \textbf{Semáforo 2}:
                \begin{itemize}
                    \item \textbf{ID}: “sem2”
                    \item \textbf{Estado}: “verde”
                    \item \textbf{Tiempo restante}: 15 segundos
                    \item \textbf{Ubicación}: (latitud, longitud)
                \end{itemize}
        \end{itemize}
    \item \textbf{Consulta (Query)}: Podemos consultar el estado actual de un semáforo específico utilizando su ID. Por ejemplo:
        \begin{itemize}
            \item “¿Cuál es el estado actual del semáforo con ID ‘sem1’?”
        \end{itemize}
    \item \textbf{Respuesta (Response)}: La aplicación recibiría la información del semáforo solicitado y podría mostrarla al usuario.
\end{enumerate}

En resumen, NGSI-LD nos permite modelar y acceder a información de contexto de manera estructurada, lo que es fundamental para construir aplicaciones inteligentes y conectadas \cite{fiware_ngsi_ld}. Las entidades y atributos definidos en NGSI-LD se utilizan para representar datos en aplicaciones FIWARE.


\section{Orion Context Broker}
El principal y único componente obligatorio de cualquier plataforma o solución desarrollada con FIWARE es el Orion Context Broker (OCB), el cual aporta una función fundamental en cualquier solución inteligente: administrar la información de contexto, consultarla y actualizarla.

El OCB permite la publicación de información de contexto por entidades, denominadas proveedores de contexto, por ejemplo los sensores, de manera que la información de contexto publicada se encuentre disponible para otras entidades, denominadas consumidores de contexto, las cuales están interesadas en procesar la información, por ejemplo, volviendo al ejemplo anterior de los semáforos (\ref{ejemplo-uso-ngsi}), una aplicación desde la que se quiere consultar el estado de los semáforos y poder así calcular la ruta más corta en un determinado momento. Los proveedores de contexto y los consumidores de contexto pueden ser cualquier aplicación o incluso otros componentes dentro de la plataforma FIWARE.

El OCB es un servidor que implementa una API que se basa en el modelo de información NGSI, por medio de la cual se pueden realizar varias operaciones: - Registrar aplicaciones de proveedores de contexto, por ejemplo: un sensor de temperatura dentro de una habitación. - Actualizar información de contexto, por ejemplo: enviar actualizaciones de la temperatura. - Ser notificado cuando surjan los cambios en la información de contexto (por ejemplo cuando la temperatura ha cambiado), o con una frecuencia determinada (por ejemplo, obtener la temperatura cada minuto). - Consultar información de contexto. Orion almacena la información de contexto actualizada desde las aplicaciones, por lo tanto, las consultas se resuelven basados en esta información.

El servidor OCB siempre está escuchando en un puerto que generalmente es el 1026. el OCB utiliza la base de datos es MongoDB para almacenar el estatus actual de las entidades, no se almacena información histórica de sus cambios. Para este propósito se debe utilizar una base de datos externa al OCB, como es el caso de Cygnus\footnote{Cygnus es un componente de la plataforma FIWARE que actúa como un conector de datos y tiene la función principal de persistir la información de contexto gestionada por el Orion Context Broker en diferentes sistemas de almacenamiento a largo plazo. Cygnus recibe notificaciones de Orion cuando hay actualizaciones en la información de contexto y las almacena en bases de datos u otros sistemas de almacenamiento seleccionados por el usuario \cite{fiware_cygnus}.} , componente el cual se estudió su implementación en este proyecto  \cite{orion}.

En esta imagen se puede apreciar el esquema de funcionamiento del OCB:
\imagen{OCBserver}{Diagrama de OCB}{.6}


\begin{comment}
En aquellos proyectos que necesiten para su comprensión y desarrollo de unos conceptos teóricos de una determinada materia o de un determinado dominio de conocimiento, debe existir un apartado que sintetice dichos conceptos.

Algunos conceptos teóricos de \LaTeX{} \footnote{Créditos a los proyectos de Álvaro López Cantero: Configurador de Presupuestos y Roberto Izquierdo Amo: PLQuiz}.

%\section{Secciones}

Las secciones se incluyen con el comando section.

\subsection{Subsecciones}

Además de secciones tenemos subsecciones.

\subsubsection{Subsubsecciones}

Y subsecciones. 


\section{Referencias}

Las referencias se incluyen en el texto usando cite~\cite{wiki:latex}. Para citar webs, artículos o libros~\cite{koza92}, si se desean citar más de uno en el mismo lugar~\cite{bortolot2005, koza92}.


\section{Imágenes}

Se pueden incluir imágenes con los comandos standard de \LaTeX, pero esta plantilla dispone de comandos propios como por ejemplo el siguiente:

\imagen{escudoInfor}{Autómata para una expresión vacía}{.5}



\section{Listas de items}

Existen tres posibilidades:

\begin{itemize}
	\item primer item.
	\item segundo item.
\end{itemize}

\begin{enumerate}
	\item primer item.
	\item segundo item.
\end{enumerate}

\begin{description}
	\item[Primer item] más información sobre el primer item.
	\item[Segundo item] más información sobre el segundo item.
\end{description}
	
\begin{itemize}
\item 
\end{itemize}

\section{Tablas}

Igualmente se pueden usar los comandos específicos de \LaTeX o bien usar alguno de los comandos de la plantilla.

\tablaSmall{Herramientas y tecnologías utilizadas en cada parte del proyecto}{l c c c c}{herramientasportipodeuso}
{ \multicolumn{1}{l}{Herramientas} & App AngularJS & API REST & BD & Memoria \\}{ 
HTML5 & X & & &\\
CSS3 & X & & &\\
BOOTSTRAP & X & & &\\
JavaScript & X & & &\\
AngularJS & X & & &\\
Bower & X & & &\\
PHP & & X & &\\
Karma + Jasmine & X & & &\\
Slim framework & & X & &\\
Idiorm & & X & &\\
Composer & & X & &\\
JSON & X & X & &\\
PhpStorm & X & X & &\\
MySQL & & & X &\\
PhpMyAdmin & & & X &\\
Git + BitBucket & X & X & X & X\\
Mik\TeX{} & & & & X\\
\TeX{}Maker & & & & X\\
Astah & & & & X\\
Balsamiq Mockups & X & & &\\
VersionOne & X & X & X & X\\
} 
\end{comment}