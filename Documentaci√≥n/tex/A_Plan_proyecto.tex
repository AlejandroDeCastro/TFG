\apendice{Plan de Proyecto Software}

\section{Introducción}

En este apartado se procede a explicar la planificación temporal del proyecto, detallando cada uno de los Sprints desde el comienzo del proyecto hasta el final del mismo. También se aborda un estudio de viabilidad del proyecto, tanto económica como legal.

\section{Planificación temporal}

Como ya se ha explicado en la sección de técnica metodológica, la distribución temporal del proyecto se ha realizado mediante \textit{Sprints} de aproximadamente dos semanas de duración.
Durante los primeros seis \textit{Sprints} se ha usado la herramienta de gestión de tareas \textit{Microsoft Planner}

\subsection{Sprint 1: Primeros pasos (23/11/2023 - 20/12/2023)}
Durante la primera reunión se barajaron diferentes propuestas de proyecto, pero hubo una que me despertó especial interés y me acabé decantando por ella. \textit{Open Data in Smart Cities}.
Se establecieron las primeras tareas, como investigar acerca de \textit{FIWARE}, descargar la plantilla de memoria, pensar en las diferentes herramientas para la organización del proyecto (Repositorio, planificador de tareas, control de versiones...). 

Durante el Sprint, se terminaron por decidir dichas herramientas, se creó el repositorio y se incorporaron en el planificador de tareas las primeras tareas para empezar con la investigación y profundización en el ámbito del proyecto.
También se comenzó con la investigación de \textit{Flask}, el \textit{framework} usado para el desarrollo de la aplicación en \textit{Python}. Como finalmente se aplazó la reunión programada para esas dos semanas, se comenzó con una pequeña demo de tratamiento de datos con \textit{Flask}.

\imagen{Sprint 1}{Sprint 1 - Tareas}{.1}

\subsection{Sprint 2: Investigación \textit{FIWARE} y desarrollo de la aplicación (20/12/2023 - 17/01/2024)}

Durante este Sprint se continuó con la investigación acerca de \textit{FIWARE}, y se trabajó en un menú para el usuario y el tratamiento de datos.
También se comenzó con una demo paralela con interfaces gráficas, para decidir si finalmente se usaba \textit{Flask} o se desarrollaba la aplicación con interfaces gráficas.

\imagen{Sprint 2}{Sprint 2 - Tareas}{.1}

\subsection{Sprint 3: Despliegue del servidor OCB (17/01/2024 - 31/01/2024)}

En esta etapa se investigó acerca del funcionamiento de \textit{Docker} y se comenzó con el despliegue del \textit{Orion Context Broker}.
También se creó un catalogo de datos anexado en la propia tarea del \textit{Planner}, el cual contiene los diferentes enlaces y los formatos que contenían. Aquí se empezó a replantear el uso de del modelo de información \textit{NGSI-LD} \ref{modelo-ngsi-ld}, debido a la falta de conjuntos con ese modelo.
También se desarrollo un \textit{Script} que transformaba ficheros en formato \textit{JSON} al modelo \textit{NGSI-LD}, dicho \textit{script} sería descartado posteriormente por su escasa eficacia y funcionalidad.

\imagen{Sprint 3}{Sprint 3 - Tareas}{.1}

\subsection{Sprint 4 (31/01/2024 - 14/02/2024)}


\imagen{Sprint 4}{Sprint 4 - Tareas}{.1}

\subsection{Sprint 5 14/02/2024 - 14/03/2024}


\imagen{Sprint 5}{Sprint 5 - Tareas}{.1}

\subsection{Sprint 6 14/03/2024 - 24/04/2024}


\imagen{Sprint 6}{Sprint 6 - Tareas}{.1}

\subsection{Sprint 7 24/04/2024 - 15/05/2024}

Cambio de Herramienta

\subsection{Sprint 8 15/05/2024 - 31/01/2024}

\subsection{Sprint 9 31/05/2024 - 14/06/2024}

\subsection{Sprint 10 14/06/2024 - 25/05/2024}

\subsection{Sprint 11 25/05/2024 - 03/05/2024}



\section{Estudio de viabilidad}

\subsection{Viabilidad económica}

\subsection{Viabilidad legal}


