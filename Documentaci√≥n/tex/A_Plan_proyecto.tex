\apendice{Plan de Proyecto Software}

\section{Introducción}

En este apartado se procede a explicar la planificación temporal del proyecto, detallando cada uno de los \textit{sprints} desde el comienzo del proyecto hasta el final del mismo. También se aborda un estudio de viabilidad del proyecto, tanto económica como legal.

\section{Planificación temporal}

Como ya se ha explicado en la sección de técnica metodológica, la distribución temporal del proyecto se ha realizado mediante \textit{Sprints} de aproximadamente dos semanas de duración.
Durante los primeros seis \textit{Sprints} se ha usado la herramienta de gestión de tareas \textit{Microsoft Planner}

\subsection{Sprint 1: Primeros pasos (23/11/2023 - 20/12/2023)}
Durante la primera reunión se barajaron diferentes propuestas de proyecto, pero hubo una que me despertó especial interés y me acabé decantando por ella. \textit{Open Data in Smart Cities}.
Se establecieron las primeras tareas, como investigar acerca de \textit{FIWARE}, descargar la plantilla de memoria, pensar en las diferentes herramientas para la organización del proyecto (Repositorio, planificador de tareas, control de versiones...). 

Durante el Sprint, se terminaron por decidir dichas herramientas, se creó el repositorio y se incorporaron en el planificador de tareas las primeras tareas para empezar con la investigación y profundización en el ámbito del proyecto.
También se comenzó con la investigación de \textit{Flask}, el \textit{framework} usado para el desarrollo de la aplicación en \textit{Python}. Como finalmente se aplazó la reunión programada para esas dos semanas, se comenzó con una pequeña demo de tratamiento de datos con \textit{Flask}.

\imagen{Sprint 1}{Sprint 1 - Tareas}

\subsection{Sprint 2: Investigación \textit{FIWARE} y desarrollo de la aplicación (20/12/2023 - 17/01/2024)}

Durante este Sprint se continuó con la investigación acerca de \textit{FIWARE}, y se trabajó en un menú para el usuario y el tratamiento de datos.
También se comenzó con una demo paralela con interfaces gráficas, para decidir si finalmente se usaba \textit{Flask} o se desarrollaba la aplicación con interfaces gráficas.

\imagen{Sprint 2}{Sprint 2 - Tareas}

\subsection{Sprint 3: Despliegue del servidor OCB (17/01/2024 - 31/01/2024)}

En esta etapa se investigó acerca del funcionamiento de \textit{Docker} y se comenzó con el despliegue del \textit{Orion Context Broker}.
También se creó un catalogo de datos anexado en la propia tarea del \textit{Planner}, el cual contiene los diferentes enlaces y los formatos que contenían. Aquí se empezó a replantear el uso de del modelo de información \textit{NGSI-LD}, debido a la falta de conjuntos con ese modelo.
También se desarrollo un \textit{Script} que transformaba ficheros en formato \textit{JSON} al modelo \textit{NGSI-LD}, dicho \textit{script} sería descartado posteriormente por su escasa eficacia y funcionalidad.

\imagen{Sprint 3}{Sprint 3 - Tareas}

\subsection{Sprint 4: Despliegue total y testeo del \textit{OCB} (31/01/2024 - 14/02/2024)}

Durante esta etapa se siguió con el desarrollo del servidor del \textit{Orion Context Broker} con el apoyo de la herramienta \textit{Postman}, con la cuál se podía ir comprobando que la base de datos está guardando bien las entidades y es capaz de devolverlas.

\imagen{Sprint 4}{Sprint 4 - Tareas}

\subsection{Sprint 5: Simulador (14/02/2024 - 14/03/2024)}

En esta etapa del proyecto afrontamos el problema de la falta de datos con el modelo \textit{NGSI-LD}, desarrollando un \textit{script} que simula un conjunto de datos con este modelo.
Este simulador representaba un conjunto de \textit{parkings}, cuyos sensores publican cada 30 segundos el estado actual de las entidades (\textit{parkings}) en la base de datos, para ser consultadas desde la aplicación.

Al no tener una caché el \textit{OCB}, no se podía guardar el estado anterior de las entidades, por lo que no se tenía un histórico de las mismas, esta funcionalidad aportaría grandes funcionalidades como análisis de patrones o predecir el estado de las entidades. Por lo tanto, durante esta etapa se investigó acerca de alguna opción que permitiese guardar históricos de las entidades.

Existe una herramienta que forma parte de la plataforma \textit{FIWARE}, llamada \textit{Cygnus}, la cual permite guardar las actualizaciones de las entidades. Esta herramienta se intentó implementar para este caso y por complejidad y poca funcionalidad (ya que es solo un simulador y se puede ir guardando un histórico antes de publicar los datos en el servidor, agilizando así la aplicación), no se terminó de implantar. Pero puede ser útil para futuras versiones.

También durante Sprint se desarrolló una demo de una plantilla de \textit{PowerBi} para dichos datos, para intentar integrarla en aplicación. El inconveniente que se encontró es que \textit{PowerBi} es un software de pago, por lo tanto, no se pudo finalmente integrar en la aplicación.


\imagen{Sprint 5}{Sprint 5 - Tareas}

\subsection{Sprint 6: Nuevas funcionalidades en la aplicación (14/03/2024 - 24/04/2024)}

Durante esta etapa la aplicación ya estaba cogiendo forma, por lo que se comenzó con los cambios estéticos, para ofrecer al usuario una mayor simpleza y pueda tener una experiencia más amigable con la aplicación.
También se añadieron algunas funcionalidades como la de los registros en la aplicación, o alguna plantilla más visual con la librería \textit{Dash}.

También se investigó e incorporó la manera de autentificar a los usuarios con un \textit{login}.

\imagen{Sprint 6}{Sprint 6 - Tareas}

\subsection{Sprint 7: (24/04/2024 - 15/05/2024)}

A partir de aquí se llevó a cabo un cambio de herramienta para la gestión de tareas. Hasta entonces se había estado usando \textit{Microsoft Planner}, pero surgió una alternativa, la cual es mucho más práctica porque permitía enlazar las tareas con los \textit{commits} y tener un mejor control de las versiones y de las tareas. Esta alternativa es ofrecida por \textit{GitHub}, nuestro repositorio y herramienta de control de versiones. Mediante los \textit{issues}, los cuales sustituyen a las tareas en el \textit{Planner} y los \textit{milestones}, los cuales representan cada uno de los \textit{sprints}, se podía lograr una mejor integración de las herramientas y una simplificación del trabajo muy destacable.

En esta etapa se desarrolló una pequeña aplicación para la gestión del servidor \textit{OCB}. Como se muestra en la imagen \ref{fig:GestorOCB}, esta aplicación permite:
\begin{itemize}
    \item \textbf{Instalar o desinstalar el servidor:} permite el la instalación o desinstalación del \textit{docker}.
    \item \textbf{Iniciar o parar el servidor:} permite iniciar o parar la ejecución del \textit{docker} de \textit{orion} y el de la \textit{MongoDB}, y eliminar su conexión.
    \item \textbf{Mostrar o eliminar:} permite mostrar o eliminar las entidades almacenadas en el servidor.
    \item \textbf{Iniciar el simulador:} inicia el simulador del modelo \textit{NGSI-LD}.
\end{itemize}

\imagen{GestorOCB}{Menú para la gestión del \textit{OCB}}
También se incorporaron nuevas funcionalidades en la aplicación como la posibilidad de añadir nuevos conjuntos de distintos formatos y una plantilla genérica para estos casos.
Por último, se trabajó en uno de los objetivos clave de la aplicación, conseguir que el usuario tenga una experiencia cómoda y sencilla con la aplicación. Para ello se investigó y se integró\textit{Bootstrap}, y también, se desarrolló una guía de usuario con \textit{GitBook}.

\imagen{Sprint 7}{Sprint 7 - Tareas}

\subsection{Sprint 8: Organización de la aplicación (15/05/2024 - 31/05/2024)}

Durante esta etapa se llevó a cabo un proceso de organización del desarrollo de la aplicación, mediante el cuál se elaboraron los diagramas de ventanas, casos de uso, interacción. Pudiendo así, enfocar la aplicación en unos requisitos determinados y simplificar algunas funcionalidades.
Durante el proceso se añadieron funcionalidades como añadir mapas en algunas vistas, diccionarios de traducciones e interpretación de datos y mejoras visuales para el usuario.

\imagen{Sprint 8}{Sprint 8 - Tareas}

\subsection{Sprint 9: Últimas funcionalidades (31/05/2024 - 14/06/2024)}

En esta etapa del proyecto se comenzó a profundizar un poco más en la redacción de la memoria y se añadieron las últimas funcionalidades en la aplicación como por ejemplo:
\begin{itemize}
    \item \textbf{Funcionalidades de administrador:} eliminar usuarios y cambiar sus roles.
    \item \textbf{Favoritos:} posibilidad de marcar conjuntos como favoritos.
    \item \textbf{\textit{Tooltip} dinámico:} en los mapas de los conjuntos, el usuario puede seleccionar que parámetros visualizar.
\end{itemize}

También se investigaron otras funcionalidades que por complejidad o falta de tiempo se terminaron descartando, como añadir filtros para los parámetros de los conjuntos o una barra de tiempo.
Otros punto a destacar en este \textit{sprint} fueron los cambios estéticos en la aplicación.

\imagen{Sprint 9}{Sprint 9 - Tareas}

\subsection{Sprint 10: Primer tramo de la recta final (14/06/2024 - 25/06/2024)}

Estando próxima la entrega, durante este \textit{sprint} se avanzó con los distintos apartados de la memoria y de los anexos.
En cuanto a la aplicación, se incorporaron mejoras visuales, como la homogeneización de la aplicación aplicando un estilo común, ayudado de \textit{Bootstrap}. Se procedió también a la revisión y mejora de algunas funcionalidades, como la gestión de registros. Y por último se incorporó un historial de errores en el cual el usuario podía dejar constancia de los distintos errores que pudieran ocurrir durante la aplicación.

\imagen{Sprint 10}{Sprint 10 - Tareas}

\subsection{Sprint 11: Últimos retoques y anexos (25/06/2024 - 03/07/2024)}

AÑADIR

\section{Estudio de viabilidad}

\subsection{Viabilidad económica}

En este apartado se va a analizar los costes y beneficios del desarrollo de este proyecto.

\subsubsection{Costes}
 
Este proyecto ha llevado más de 500 horas de trabajo:\begin{itemize}
    \item \textbf{480 horas} de trabajo por parte del alumno.
    \item \textbf{20 horas} de reuniones con los tutores.
\end{itemize}  

Para extrapolar estas cifras a un proyecto real, se identificarán a los tutores como \textit{Project Mangers} y al alumno como el resto de personal que formaría parte de un equipo de desarrollo (analista, diseñador, programador...).

Un \textit{Project Manger} en España cobra de media 40.000€ anuales\footnote{\url{https://www.glassdoor.es/Sueldos/project-manager-sueldo-SRCH_KO0,15.htm}}, teniendo en cuenta que la jornada laboral en España normalmente es de 8 horas al día durante 5 días a la semana, el sueldo por hora sería de 19,23€\footnote{Conversión realizada con \url{https://convertir-sueldo-hora-ano.appspot.com}}.
Si multiplicamos las 20 horas totales de cada \textit{Project Manger} por el salario por hora medio en España de un \textit{Project Manger} por los dos \textit{Project Mangers} del proyecto, obtenemos un total de 769,2€

$$20 \, \text{horas} \times 19,23 \, \text{€/hora} \times 2 \, \text{personas} = \text{769,2€}$$

El sueldo medio de un desarrollador en España es de 20.000€ anuales\footnote{\url{https://www.hackaboss.com/blog/cuanto-gana-programador-salario-espana}}, volviendo a hacer la conversión a euros la hora, sería 9,62€ por hora\footnote{Conversión realizada con \url{https://convertir-sueldo-hora-ano.appspot.com}}.
Si multiplicamos las 480 horas totales del proyecto por el salario por hora medio en España de un desarrollador, obtenemos un total de 4.617,60€

$$480 \, \text{horas} \times 9,62 \, \text{€/hora} = \text{4.617,60€}$$

En total el coste de personal sería de 5.386,8€
$$769,2 \, \text{€} + 4.617,60 \, \text{€} = \text{5.386,8€}$$

El equipo utilizado es un ordenador Lenovo Legion Y520 con un procesador Intel® Core™ i7-7700HQ de 3.8GHz con 8 GB de RAM, valorado en 799,99€\footnote{\url{https://www.mediamarkt.es/es/product/_portátil-gaming-lenovo-legion-y520-15ikbn-15-6-fhd-intel®-core™-i7-7700hq-8gb-1tb-gtx1050-fdos-1401315.html}}.

En cuanto a \textit{software}, todas las herramientas son gratuitas. Solo se ha usado de pago la herramienta \LaTeX{}, que aunque es gratuita, el último mes se ha decidido probar por 8€ su versión Pro, para poder mejorar la conectividad con \textit{GitHub} durante la redacción de la memoria.

En total el coste del \textit{hardware} y \textit{software} sería de 5.386,8€
$$799,99 \, \text{€} + 8 \, \text{€} = \text{807,99€}$$

Si se suma el coste de personal con el de \textit{hardware} y \textit{software}, sale un total de 6.194,79€
$$5.386,8 \, \text{€} + 807,99 \, \text{€} = \text{6.194,79€}$$

\subsubsection{Beneficios}

El proyecto está enfocado a acercar el acceso a los datos en tiempo real a la población de una manera no lucrativa, es por ello que la aplicación es totalmente gratuita.
Una opción que se podría incluir para cubrir los gastos del mantenimiento de la aplicación puede ser incluir un apartado de donaciones, para que los usuarios que quieran contribuir puedan hacerlo de manera altruista.

Por otro lado, si fuese necesario se podría añadir anuncios de manera que generen algunos beneficios para el mantenimiento de la aplicación, pero en ningún caso se plantean fines lucrativos con la aplicación.

\subsection{Viabilidad legal}


