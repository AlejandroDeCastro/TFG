\apendice{Especificación de diseño}

\section{Introducción}
En este anexo se define la resolución de los objetivos y especificaciones de los anexos anteriores.

Su estructura es la siguiente:
\begin{itemize}
    \item \textbf{Diseño de datos:} se detalla el modelado de datos de la aplicación mediante el diagrama Entidad-Relación \ref{e-r}, el modelo relacional \ref{modelo-relacional} y el diccionario de datos \ref{diccionario-de-datos} de la aplicación.
    \item \textbf{Diseño procedimental:} se detalla mediante diagramas de secuencia las distintas funcionalidades de la aplicación.
    \item \textbf{Diseño arquitectónico:} se muestra un diagrama de la arquitectura de la aplicación.
    \item \textbf{Diseño gráfico:} se muestra el diagrama de interfaces desarrollado como guía para el producto final.
\end{itemize}

\section{Diseño de datos}
\subsection{Diagrama Entidad-Relación}\label{e-r}

Un diagrama entidad-relación (ERD) describe los datos o información de un sistema mediante modelado de sus entidades, atributos y las relaciones entre esas entidades. \cite{Chen1976}

Como se puede ver en el diagrama \ref{fig:E_R}, existen 4 entidades:
\begin{itemize}
    \item \textbf{Usuarios:} poseen identificador, usuario, contraseña, nombre completo, favoritos y rol.
    \item \textbf{Datos:} poseen identificador, ciudad, característica, enlace, formato y periodicidad.
    \item \textbf{Registros:} poseen identificador, ciudad, característica, formato, periodicidad y \textit{pid}.
    \item \textbf{Traducciones:} poseen identificador, usuario, contraseña, nombre completo, favoritos y rol.
\end{itemize}

Usuarios se relaciona con Datos con \textit{id usuario}, guardado en Datos. Un usuario puede añadir ninguno o N conjuntos de datos, por lo tanto, cada conjunto de datos almacenado debe tener como mínimo y máximo un \textit{id usuario} para identificar quién guardó ese conjunto.

Usuarios se relaciona con Registros con \textit{id usuario}, guardado en Registros. Un usuario puede tener ninguno o N registros para él, por lo tanto, cada registro almacenado debe tener como mínimo y máximo un \textit{id usuario} para identificar para quién es dicho registro.

Traducciones no tiene relación con ninguna tabla, ya que solo pueden escribir administradores en esa tabla, por lo tanto, no es interesante guardar el \textit{id usuario} que añadió la traducción en cuestión, ya que los administradores son un número muy reducido de usuarios y de confianza. Y el número de traducciones es muy elevado por lo que se guardaría información repetida e irrelevante de manera muy recurrente.

\imagen{E_R}{Diagrama Entidad-Relación}


\subsection{Modelo Relacional}\label{modelo-relacional}

Un modelo relacional es una forma de estructurar y consultar datos en una base de datos. En este modelo, los datos se organizan en tablas (o relaciones), donde cada tabla está compuesta por filas (tuplas) y columnas (atributos). \cite{Codd1970}

El modelo relacional de esta aplicación lo podemos observar en la imagen \ref{fig:ModeloRelacional}

\imagen{ModeloRelacional}{Modelo Relacional}


\subsection{Diccionario de datos}\label{diccionario-de-datos}

Un diccionario de datos es una herramienta crucial en la gestión de bases de datos que almacena definiciones y descripciones de los elementos de datos utilizados en el sistema. Proporciona una visión centralizada y coherente de los datos, incluyendo detalles sobre los nombres, tipos, formatos, y relaciones entre los elementos de datos. \cite{Kent1983}

La tabla~\ref{tabla-dic-usuarios} contiene la información de la tabla de Usuarios.

\begin{table}
	\scalebox{0.80}{
		\begin{tabular}{@{}p{10em} p{6em} p{6em} p{20em}@{}}
			\toprule
			\textbf{Nombre} & \textbf{Tipo} & \textbf{Columna} & \textbf{Descripción}\\
			\midrule
			\texttt{\textbf{\underline{id}}} & \texttt{SMALLINT(3)} & \texttt{\textbf{\underline{PK}}} & Identificador. Generado automáticamente. \\
			\texttt{usuario} &  \texttt{VARCHAR(20)} & \texttt{NOT NULL} & Nombre de usuario proporcionado en el registro.\\
			\texttt{contraseña} &  \texttt{CHAR(255)} & \texttt{NOT NULL} & Contraseña del usuario cifrada.\\
			\texttt{nombre\_completo} & \texttt{VARCHAR(50)} & \texttt{NOT NULL} & Nombre completo del usuario.\\
			\texttt{favoritos} & \texttt{text} & \texttt{NOT NULL} & Lista de conjuntos favoritos del usuario separada por comas. \\
			\texttt{rol} & \texttt{VARCHAR(15)} & \texttt{NOT NULL} & Rol del usuario. \\
			\bottomrule
		\end{tabular}
	}
	\caption{Diccionario de Datos - Tabla Usuarios.}
    \label{tabla-dic-usuarios}
\end{table}

La tabla~\ref{tabla-dic-datos} contiene la información de la tabla de Datos.
\begin{table}
	\scalebox{0.80}{
		\begin{tabular}{@{}p{10em} p{6em} p{6em} p{20em}@{}}
			\toprule
			\textbf{Nombre} & \textbf{Tipo} & \textbf{Columna} & \textbf{Descripción}\\
			\midrule
			\texttt{\textbf{\underline{id}}} & \texttt{SMALLINT(6)} & \texttt{\textbf{\underline{PK}}} & Identificador. Generado automáticamente. \\
			\texttt{ciudad} &  \texttt{VARCHAR(25)} & \texttt{NOT NULL} & Ciudad en la que se encuentra el conjunto de datos.\\
			\texttt{característica} &  \texttt{VARCHAR(100)} & \texttt{NOT NULL} & Nombre del conjunto de datos.\\
			\texttt{enlace} &  \texttt{VARCHAR(10000)} & \texttt{NOT NULL} & Enlace al conjunto de datos.\\
			\texttt{id\_usuario} & \texttt{SMALLINT(3)} & \texttt{NOT NULL} & Identificador del usuario que ha añadido el conjunto.\\
			\texttt{formato} & \texttt{VARCHAR(8)} & \texttt{NOT NULL} & Formato del conjunto de datos. \\
			\texttt{periodicidad} & \texttt{INTEGER(250)} & \texttt{NOT NULL} & Periodicidad de actualización del conjunto de datos en segundos. \\
			\bottomrule
		\end{tabular}
	}
	\caption{Diccionario de Datos - Tabla Datos.}
    \label{tabla-dic-datos}
\end{table}

La tabla~\ref{tabla-dic-registros} contiene la información de la tabla de Registros.
\begin{table}
	\scalebox{0.80}{
		\begin{tabular}{@{}p{10em} p{6em} p{6em} p{20em}@{}}
			\toprule
			\textbf{Nombre} & \textbf{Tipo} & \textbf{Columna} & \textbf{Descripción}\\
			\midrule
			\texttt{\textbf{\underline{id}}} & \texttt{SMALLINT(3)} & \texttt{\textbf{\underline{PK}}} & Identificador. Generado automáticamente. \\
			\texttt{ciudad} &  \texttt{VARCHAR(25)} & \texttt{NOT NULL} & Ciudad en la que se encuentra el conjunto de datos.\\
			\texttt{característica} &  \texttt{VARCHAR(100)} & \texttt{NOT NULL} & Nombre del conjunto de datos.\\
			\texttt{id\_usuario} & \texttt{SMALLINT(3)} & \texttt{NOT NULL} & Identificador del usuario al que pertenece el registro.\\
			\texttt{formato} & \texttt{VARCHAR(8)} & \texttt{NOT NULL} & Formato del conjunto de datos. \\
			\texttt{periodicidad} & \texttt{INTEGER(250)} & \texttt{NOT NULL} & Periodicidad de captura de datos. \\
            \texttt{pid} &  \texttt{INTEGER(6)} & \texttt{NOT NULL} & Identificación del proceso que realiza captura de datos.\\
			\bottomrule
		\end{tabular}
	}
	\caption{Diccionario de Datos - Tabla Registros.}
    \label{tabla-dic-registros}
\end{table}

La tabla~\ref{tabla-dic-traducciones} contiene la información de la tabla de Traducciones.
\begin{table}
	\scalebox{0.80}{
		\begin{tabular}{@{}p{10em} p{6em} p{6em} p{20em}@{}}
			\toprule
			\textbf{Nombre} & \textbf{Tipo} & \textbf{Columna} & \textbf{Descripción}\\
			\midrule
			\texttt{\textbf{\underline{id}}} & \texttt{SMALLINT(3)} & \texttt{\textbf{\underline{PK}}} & Identificador. Generado automáticamente. \\
			\texttt{original} &  \texttt{VARCHAR(100)} & \texttt{NOT NULL} & Palabra o frase a traducir.\\
            \texttt{traducción} &  \texttt{VARCHAR(100)} & \texttt{NOT NULL} & Palabra o frase traducida.\\
			\bottomrule
		\end{tabular}
	}
	\caption{Diccionario de Datos - Tabla Traducciones.}
    \label{tabla-dic-traducciones}
\end{table}

\section{Diseño procedimental}

Un diagrama de secuencias es un tipo de diagrama de interacción. Este diagrama se utiliza para representar cómo los objetos interactúan en un proceso específico a lo largo del tiempo, mostrando la secuencia de mensajes intercambiados entre los objetos para llevar a cabo una funcionalidad particular. \cite{Booch1998}

Gracias al diagrama de secuencias se puede describir en detalle cómo se lleva a cabo una funcionalidad específica. Es por ello, por lo que se han elaborado los siguientes diagramas de secuencia, los cuales han servido de guía para el desarrollo de la aplicación.

\subsection{Crear cuenta}
\imagen{UC-1}{Procedimiento de crear una cuenta}

\subsection{Iniciar sesión}
\imagen{UC-2}{Procedimiento de iniciar sesión}

\subsection{Visualizar conjunto}
\imagen{UC-3}{Procedimiento de visualizar un conjunto}

\subsection{Añadir conjunto}
\imagen{UC-4}{Procedimiento de añadir un conjunto}

\subsection{Mostrar registros}
\imagen{UC-5}{Procedimiento de mostrar los registros}

\subsection{Añadir registro}
\imagen{UC-5.1}{Procedimiento de añadir un registro}

\subsection{Descargar registro}
\imagen{UC-5.2}{Procedimiento de descargar registro}

\subsection{Eliminar registro}
\imagen{UC-5.3}{Procedimiento de eliminar registro}

\subsection{Eliminar conjunto}
\imagen{UC-10}{Procedimiento de eliminar un conjunto}

\subsection{Añadir traducciones}
\imagen{UC-11.1}{Procedimiento de añadir una traducción}


\section{Diseño arquitectónico}

\subsection{Estructura del \textit{software} del proyecto}
\imagen{Estructura}{Estructura del proyecto}

\subsection{Arquitectura de la aplicación}
\imagen{Diseño arquitectónico}{Diseño arquitectónico de la aplicación}

\section{Diseño gráfico}

Se ha elaborado un

\subsection{Crear cuenta}
\imagen{crear_cuenta_pantalla}{Crear cuenta - Pantalla}

\subsection{Login}
\imagen{login_pantalla}{Login - Pantalla}

\subsection{Home - Usuario}
\imagen{home_pantalla}{Home usuario - Pantalla}

\subsection{Home - Administrador}
\imagen{home_admin_pantalla}{Home administrador - Pantalla}

\subsection{Conjuntos}
\imagen{conjuntos_pantalla}{Conjuntos - Pantalla}

\subsection{Registros}
\imagen{registros_pantalla}{Registros - Pantalla}

\subsection{Conjunto}
\imagen{conjunto_pantalla}{Conjunto - Pantalla}

\subsection{Ayuda}
\imagen{ayuda_pantalla}{Ayuda - Pantalla}
