\apendice{Especificación de Requisitos}

\section{Introducción}

En este apartado se van a explicar los distintos requisitos de la aplicación, marcados principalmente por los objetivos establecidos \ref{objetivos-generales}.
Estos requisitos son la base de la aplicación, ya que actúan como guía después de la etapa de análisis para el correcto desarrollo de la aplicación con un enfoque claro y unos objetivos concretos.

El anexo está estructurado de la siguiente forma:
\begin{itemize}
    \item Objetivos generales: contiene los objetivos generales del proyecto.
    \item Catálogo de requisitos: se detallan los requisitos funcionales y no funcionales de la aplicación 
    \item Especificación de requisitos: se muestra una visión general de los casos de uso del sistema y posteriormente se detallan uno a uno.
\end{itemize}

\section{Objetivos generales}\label{objetivos-generales}

Los objetivos generales del proyecto son los siguientes:

\begin{itemize}
    \item Desarrollar una aplicación que permita a cualquier usuario el acceso a los distintos conjuntos de datos, que los servicios de \textit{IT} hayan publicado en abierto, de manera visual, personalizada y precisa.
    \item Ofrecer al usuario una experiencia intuitiva y cómoda con la aplicación.
    \item Permitir al usuario la posibilidad de guardar un histórico de los conjuntos que desee para un posterior análisis de los mismos.
    \item Contribuir con el crecimiento de la plataforma \textit{FIWARE}.

\end{itemize}

\section{Catálogo de requisitos}\label{catalogo-de-requisitos}

A continuación se detallan los requisitos funcionales y no funcionales de la aplicación.

\subsection{\textbf{Requisitos funcionales}}

\begin{itemize}
    \item \textbf{RF-1 Inicio de sesión:} El Sistema debe permitir al usuario iniciar sesión introduciendo su usuario y su contraseña.
    \begin{itemize}
        \item \textbf{RF-1.1 Privacidad de contraseña:} Debe poder ocultarse o mostrarse la contraseña.
        \item \textbf{RF-1.2 Contraseña incorrecta:} Debe notificarse al usuario si la contraseña no es correcta.
        \item \textbf{RF-1.3 Usuario inexistente:} Debe notificarse al usuario si el nombre introducido no se encuentra en el Sistema.
    \end{itemize}

    \item \textbf{RF-2 Crear una cuenta:} El Sistema debe permitir al usuario crear una cuenta introduciendo un nombre de usuario, su nombre completo y una contraseña.
    \begin{itemize}
        \item \textbf{RF-2.1 Verificación de usuario:} Debe notificarse al usuario si ya existe un usuario con ese nombre de usuario en el Sistema.
        \item \textbf{RF-2.2 Privacidad de incorrecta:} Debe poder ocultarse o mostrarse la contraseña.
        \item \textbf{RF-2.3 Verificación de contraseña:} Debe comprobarse que las contraseñas coincidad y en caso contrario notificarlo al usuario.
    \end{itemize}

    \item \textbf{RF-3 Elegir conjuntos favoritos:} El usuario debe poder marcar o desmarcar cualquier conjunto como favorito desde la vista de dicho conjunto.
    \begin{itemize}
        \item \textbf{RF-3.1 Acceso rápido:} Los conjuntos marcados como favoritos aparecerán en la pantalla de inicio con el objetivo de poder acceder a ellos de manera rápida.
    \end{itemize}

    \item \textbf{RF-4 Gestión conjuntos:} El usuario debe poder añadir conjuntos de datos en distintos formatos a la aplicación para que el resto de usuarios puedan consultarlos.
    \begin{itemize}
        \item \textbf{RF-4.1 Campos a introducir:} El usuario debe introducir obligatoriamente los campos de ciudad, conjunto, formato y enlace, y opcionalmente la periodicidad con la que se actualiza dicho conjunto.
        \item \textbf{RF-4.2 Eliminación de conjuntos:} El administrador debe poder eliminar los conjuntos que considere pertinentes.
    \end{itemize}

    \item \textbf{RF-5 Gestión de registros:} El Sistema debe ofrecer al usuario de guardar un histórico de los conjuntos con una periodicidad indicada por el usuario, de manera que el usuario pueda descargarlos cuando desee.
    \begin{itemize}
        \item \textbf{RF-5.1 Registros repetidos:} La periodicidad no puede ser menor a la indicada en origen del conjunto de datos.
        \item \textbf{RF-5.2 Eliminación de registros:} El usuario debe poder eliminar el registro de un conjunto de datos.
    \end{itemize}

    \item \textbf{RF-6 Gestión de usuarios:} los usuarios con el rol de administrador deben poder gestionar el resto de usuarios.
    \begin{itemize}
        \item \textbf{RF-6.1 Cambio de rol:} los administradores deben poder cambiar el rol de cualquier usuario.
        \item \textbf{RF-6.2 Eliminar usuarios:} los administradores deben poder eliminar cualquier cuenta, menos la suya.
    \end{itemize}

    \item \textbf{RF-7 Traducción de campos:} los campos de los conjuntos de datos deben poder ser traducidos para una experiencia más cómoda del usuario.
    \begin{itemize}
        \item \textbf{RF-7.1 Gestionar traducciones:} los administradores deben poder añadir o eliminar traducciones de la aplicación, las cuales se aplican a todos los conjuntos.
    \end{itemize}

    \item \textbf{RF-8 Visualizar conjuntos de datos en tiempo real:} el usuario debe poder visualizar un conjunto de datos de una forma funcional y cómoda.
    \begin{itemize}
        \item \textbf{RF-8.1 Ordenación de entidades:} los usuarios deben poder ordenar alfabéticamente ascendente y descendente las distintas entidades mostradas en función de los parámetros de las mismas.
    \end{itemize}

    \item \textbf{RF-9 Reporte de errores:} el usuario debe poder reportar los errores que le impidan tener un uso adecuado de la aplicación.
    
\end{itemize}

\subsection{\textbf{Requisitos no funcionales}}

\begin{itemize}
\tightlist
\item
    \textbf{RNF-1 Usabilidad:} la aplicación debe ofrecer una experiencia cómoda para el usuario, siendo lo más intuitiva y fácil de entender posible para el usuario.
\item
    \textbf{RNF-2 Compatibilidad:} la aplicación debe ser compatible con cualquier navegador en cualquier dispositivo.
\item
    \textbf{RNF-3 Disponibilidad:} la aplicación debe estar operativa en todo momento para los usar.
\item
    \textbf{RNF-4 Mantenibilidad y Escalabilidad}: la aplicación debe poder sen mantenible y fácilmente modificada por un desarrollador para aplicar futuras mejoras.
\item
    \textbf{RNF-5 Rendimiento}: la navegación por la aplicación debe ser ágil y los tiempos de carga de los conjuntos de datos deben ser muy reducidos.
\item
    \textbf{RNF-6 Persistencia}: los datos manejados relacionados a los usuarios deben de ser almacenados correctamente y de forma segura.
\item
    \textbf{RNF-7 Robustez:} la aplicación debe ser robusta y tener una respuesta y control a todos los errores para garantizar el correcto funcionamiento de la misma de forma continuada
\end{itemize}

\section{Especificación de requisitos}

Una vez detallados los requisitos funcionales de la aplicación, se detallan a continuación los casos de uso que proporcionan un contexto detallado a los requisitos del apartado anterior. \ref{catalogo-de-requisitos}

Los actores definidos en este sistema son:
\begin{itemize}
\tightlist
\item
    \textbf{Usuario:} cualquier usuario que se registre en la aplicación.
\item
    \textbf{Administrador:} usuario registrado en la aplicación con permisos de administrador, este rol solo puede ser asignado o denegado por otro administrador.
\end{itemize}


% Caso de Uso 1 -> Consultar Experimentos.
\begin{table}[p]
	\centering
	\begin{tabularx}{\linewidth}{ p{0.21\columnwidth} p{0.71\columnwidth} }
		\toprule
		\textbf{CU-1}    & \textbf{Ejemplo de caso de uso}\\
		\toprule
		\textbf{Versión}              & 1.0    \\
		\textbf{Autor}                & {Alejandro de Castro} \\
		\textbf{Requisitos asociados} & RF-xx, RF-xx \\
		\textbf{Descripción}          & La descripción del CU \\
		\textbf{Precondición}         & Precondiciones (podría haber más de una) \\
		\textbf{Acciones}             &
		\begin{enumerate}
			\def\labelenumi{\arabic{enumi}.}
			\tightlist
			\item Pasos del CU
			\item Pasos del CU (añadir tantos como sean necesarios)
		\end{enumerate}\\
		\textbf{Postcondición}        & Postcondiciones (podría haber más de una) \\
		\textbf{Excepciones}          & Excepciones \\
		\textbf{Importancia}          & Alta o Media o Baja... \\
		\bottomrule
	\end{tabularx}
	\caption{CU-1 Nombre del caso de uso.}
\end{table}