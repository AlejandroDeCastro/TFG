\apendice{Especificación de Requisitos}

\section{Introducción}

Una muestra de cómo podría ser una tabla de casos de uso:

% Caso de Uso 1 -> Consultar Experimentos.
\begin{table}[p]
	\centering
	\begin{tabularx}{\linewidth}{ p{0.21\columnwidth} p{0.71\columnwidth} }
		\toprule
		\textbf{CU-1}    & \textbf{Ejemplo de caso de uso}\\
		\toprule
		\textbf{Versión}              & 1.0    \\
		\textbf{Autor}                & {Alejandro de Castro} \\
		\textbf{Requisitos asociados} & RF-xx, RF-xx \\
		\textbf{Descripción}          & La descripción del CU \\
		\textbf{Precondición}         & Precondiciones (podría haber más de una) \\
		\textbf{Acciones}             &
		\begin{enumerate}
			\def\labelenumi{\arabic{enumi}.}
			\tightlist
			\item Pasos del CU
			\item Pasos del CU (añadir tantos como sean necesarios)
		\end{enumerate}\\
		\textbf{Postcondición}        & Postcondiciones (podría haber más de una) \\
		\textbf{Excepciones}          & Excepciones \\
		\textbf{Importancia}          & Alta o Media o Baja... \\
		\bottomrule
	\end{tabularx}
	\caption{CU-1 Nombre del caso de uso.}
\end{table}

\section{Objetivos generales}

\section{Catálogo de requisitos}

\subsection{\textbf{Requisitos funcionales}}

\begin{itemize}
    \item \textbf{RF-1 Inicio de sesión:} El Sistema debe permitir al usuario iniciar sesión introduciendo su usuario y su contraseña.
    \begin{itemize}
        \item \textbf{RF-1.1 Privacidad de contraseña:} Debe poder ocultarse o mostrarse la contraseña.
        \item \textbf{RF-1.2 Contraseña incorrecta:} Debe notificarse al usuario si la contraseña no es correcta.
        \item \textbf{RF-1.3 Usuario inexistente:} Debe notificarse al usuario si el nombre introducido no se encuentra en el Sistema.
    \end{itemize}

    \item \textbf{RF-2 Crear una cuenta:} El Sistema debe permitir al usuario crear una cuenta introduciendo un nombre de usuario, su nombre completo y una contraseña.
    \begin{itemize}
        \item \textbf{RF-2.1 Verificación de usuario:} Debe notificarse al usuario si ya existe un usuario con ese nombre de usuario en el Sistema.
        \item \textbf{RF-2.2 Privacidad de incorrecta:} Debe poder ocultarse o mostrarse la contraseña.
        \item \textbf{RF-2.3 Verificación de contraseña:} Debe comprobarse que las contraseñas coincidad y en caso contrario notificarlo al usuario.
    \end{itemize}

    \item \textbf{RF-3 Elegir conjuntos favoritos:} El usuario debe poder marcar o desmarcar cualquier conjunto como favorito desde la vista de dicho conjunto.
    \begin{itemize}
        \item \textbf{RF-3.1 Acceso rápido:} Los conjuntos marcados como favoritos aparecerán en la pantalla de inicio con el objetivo de poder acceder a ellos de manera rápida.
    \end{itemize}
\end{itemize}


\section{Especificación de requisitos}


