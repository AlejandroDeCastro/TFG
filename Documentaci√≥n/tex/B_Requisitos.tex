\apendice{Especificación de Requisitos}

\section{Introducción}

En este apartado se van a explicar los distintos requisitos de la aplicación, marcados principalmente por los objetivos establecidos \ref{objetivos-generales}.
Estos requisitos son la base de la aplicación, ya que actúan como guía después de la etapa de análisis para el correcto desarrollo de la aplicación con un enfoque claro y unos objetivos concretos.

El anexo está estructurado de la siguiente forma:
\begin{itemize}
    \item Objetivos generales: contiene los objetivos generales del proyecto.
    \item Catálogo de requisitos: se detallan los requisitos funcionales y no funcionales de la aplicación 
    \item Especificación de requisitos: se muestra una visión general de los casos de uso del sistema y posteriormente se detallan uno a uno.
\end{itemize}

\section{Objetivos generales}\label{objetivos-generales}

Los objetivos generales del proyecto son los siguientes:

\begin{itemize}
    \item Desarrollar una aplicación que permita a cualquier usuario el acceso a los distintos conjuntos de datos, que los servicios públicos de datos abiertos de diferentes ayuntamientos españoles hayan publicado en abierto, de manera visual, personalizada y precisa.
    \item Ofrecer al usuario una experiencia intuitiva y cómoda con la aplicación.
    \item Permitir al usuario la posibilidad de guardar un histórico de los conjuntos que desee para un posterior análisis de los mismos.
    \item Contribuir con el crecimiento de la plataforma \textit{Fiware} mediante el desarrollo de una aplicación con la implementación de los estándares de dicha plataforma.

\end{itemize}

\section{Catálogo de requisitos}\label{catalogo-de-requisitos}

A continuación se detallan los requisitos funcionales y no funcionales de la aplicación.

\subsection{\textbf{Requisitos funcionales}}

\begin{itemize}
    \item \textbf{RF-1 Inicio de sesión:} El Sistema debe permitir al usuario iniciar sesión introduciendo su usuario y su contraseña.
    \begin{itemize}
        \item \textbf{RF-1.1 Privacidad de contraseña:} Debe poder ocultarse o mostrarse la contraseña.
        \item \textbf{RF-1.2 Contraseña incorrecta:} Debe notificarse al usuario si la contraseña no es correcta.
        \item \textbf{RF-1.3 Usuario inexistente:} Debe notificarse al usuario si el nombre introducido no se encuentra en el Sistema.
    \end{itemize}

    \item \textbf{RF-2 Crear una cuenta:} El Sistema debe permitir al usuario crear una cuenta introduciendo un nombre de usuario, su nombre completo y una contraseña.
    \begin{itemize}
        \item \textbf{RF-2.1 Verificación de usuario:} Debe notificarse al usuario si ya existe un usuario con ese nombre de usuario en el Sistema.
        \item \textbf{RF-2.2 Privacidad de incorrecta:} Debe poder ocultarse o mostrarse la contraseña.
        \item \textbf{RF-2.3 Verificación de contraseña:} Debe comprobarse que las contraseñas coincidad y en caso contrario notificarlo al usuario.
    \end{itemize}

    \item \textbf{RF-3 Elegir conjuntos favoritos:} El usuario debe poder marcar o desmarcar cualquier conjunto como favorito desde la vista de dicho conjunto.
    \begin{itemize}
        \item \textbf{RF-3.1 Acceso rápido:} Los conjuntos marcados como favoritos aparecerán en la pantalla de inicio con el objetivo de poder acceder a ellos de manera rápida.
    \end{itemize}

    \item \textbf{RF-4 Gestión conjuntos:} El usuario debe poder añadir conjuntos de datos en distintos formatos a la aplicación para que el resto de usuarios puedan consultarlos.
    \begin{itemize}
        \item \textbf{RF-4.1 Campos a introducir:} El usuario debe introducir obligatoriamente los campos de ciudad, conjunto, formato y enlace, y opcionalmente la periodicidad con la que se actualiza dicho conjunto.
        \item \textbf{RF-4.2 Eliminación de conjuntos:} El administrador debe poder eliminar los conjuntos que considere pertinentes.
    \end{itemize}

    \item \textbf{RF-5 Gestión de registros:} El Sistema debe ofrecer al usuario la posibilidad de guardar un histórico de los conjuntos con una periodicidad indicada por el usuario.
    \begin{itemize}
        \item \textbf{RF-5.1 Registros repetidos:} La periodicidad no puede ser menor a la indicada en origen del conjunto de datos.
        \item \textbf{RF-5.2 Descarga de registros:} El usuario debe poder descargar un conjunto de datos de los que esté guardando.
        \item \textbf{RF-5.3 Eliminación de registros:} El usuario debe poder eliminar el registro de un conjunto de datos.
    \end{itemize}

    \item \textbf{RF-6 Gestión de usuarios:} los usuarios con el rol de administrador deben poder gestionar el resto de usuarios.
    \begin{itemize}
        \item \textbf{RF-6.1 Cambio de rol:} los administradores deben poder cambiar el rol de cualquier usuario.
        \item \textbf{RF-6.2 Eliminar usuarios:} los administradores deben poder eliminar cualquier cuenta, menos la suya.
    \end{itemize}

    \item \textbf{RF-7 Traducción de campos:} los campos de los conjuntos de datos deben poder ser traducidos para una experiencia más cómoda del usuario.
    \begin{itemize}
        \item \textbf{RF-7.1 Gestionar traducciones:} los administradores deben poder añadir o eliminar traducciones de la aplicación, las cuales se aplican a todos los conjuntos.
    \end{itemize}

    \item \textbf{RF-8 Visualizar conjuntos de datos en tiempo real:} el usuario debe poder visualizar un conjunto de datos de una forma funcional y cómoda.
    \begin{itemize}
        \item \textbf{RF-8.1 Ordenación de entidades:} los usuarios deben poder ordenar alfabéticamente ascendente y descendente las distintas entidades mostradas en función de los parámetros de las mismas.
        \item \textbf{RF-8.2 Modificación del \textit{tooltip} del mapa:} los usuarios deben poder indicar que parámetros desean mostrar u ocultar en el mapa.
    \end{itemize}

    \item \textbf{RF-9 Reporte de errores:} el usuario debe poder reportar los errores que le impidan tener un uso adecuado de la aplicación.

    \item \textbf{RF-10 Mostrar ayuda:} el usuario debe poder consultar una pantalla de ayuda con instrucciones de ayuda y contacto con los administradores.
    
\end{itemize}

\subsection{\textbf{Requisitos no funcionales}}

\begin{itemize}
\tightlist
\item
    \textbf{RNF-1 Usabilidad:} la aplicación debe ofrecer una experiencia cómoda para el usuario, siendo lo más intuitiva y fácil de entender posible para el usuario.
\item
    \textbf{RNF-2 Compatibilidad:} la aplicación debe ser compatible con cualquier navegador en cualquier dispositivo.
\item
    \textbf{RNF-3 Disponibilidad:} la aplicación debe estar operativa en todo momento para los usar.
\item
    \textbf{RNF-4 Mantenibilidad y Escalabilidad}: la aplicación debe poder sen mantenible y fácilmente modificada por un desarrollador para aplicar futuras mejoras.
\item
    \textbf{RNF-5 Rendimiento}: la navegación por la aplicación debe ser ágil y los tiempos de carga de los conjuntos de datos deben ser muy reducidos.
\item
    \textbf{RNF-6 Persistencia}: los datos manejados relacionados a los usuarios deben de ser almacenados correctamente y de forma segura.
\item
    \textbf{RNF-7 Robustez:} la aplicación debe ser robusta y tener una respuesta y control a todos los errores para garantizar el correcto funcionamiento de la misma de forma continuada
\end{itemize}

\section{Especificación de requisitos}

Una vez detallados los requisitos funcionales de la aplicación, se detallan a continuación los casos de uso que proporcionan un contexto detallado a los requisitos del apartado anterior. \ref{catalogo-de-requisitos}

\subsection{Actores}
Los actores definidos en este sistema son:
\begin{itemize}
\tightlist
\item
    \textbf{Usuario:} cualquier usuario que se registre en la aplicación.
\item
    \textbf{Administrador:} usuario registrado en la aplicación con permisos de administrador, este rol solo puede ser asignado o denegado por otro administrador.
\end{itemize}

\subsection{Diagrama de casos de uso}
\imagen{Diagrama casos de uso}{Diagrama de casos de uso}

\subsection{Casos de uso}
% Caso de Uso 1 -> Crear una cuenta.
\begin{table}[p]
	\centering
	\begin{tabularx}{\linewidth}{ p{0.21\columnwidth} p{0.71\columnwidth} }
		\toprule
		\textbf{CU-1}    & \textbf{Crear una cuenta}\\
		\toprule
		\textbf{Versión}              & 1.0    \\
		\textbf{Autor}                & {Alejandro de Castro} \\
		\textbf{Requisitos asociados} & RF-2, RF-xx \\
		\textbf{Descripción}          & La descripción del CU \\
		\textbf{Precondición}         & Precondiciones (podría haber más de una) \\
		\textbf{Acciones}             &
		\begin{enumerate}
			\def\labelenumi{\arabic{enumi}.}
			\tightlist
			\item Pasos del CU
			\item Pasos del CU (añadir tantos como sean necesarios)
		\end{enumerate}\\
		\textbf{Postcondición}        & Postcondiciones (podría haber más de una) \\
		\textbf{Excepciones}          & Excepciones \\
		\textbf{Importancia}          & Alta o Media o Baja... \\
		\bottomrule
	\end{tabularx}
	\caption{CU-1 Nombre del caso de uso.}
\end{table}


% Caso de Uso 1 -> Crear cuenta.
\begin{table}[p]
	\centering
	\begin{tabularx}{\linewidth}{ p{0.21\columnwidth} p{0.71\columnwidth} }
		\toprule
		\textbf{CU-1}    & \textbf{Crear cuenta}\\
		\toprule
		\textbf{Versión}              & 1.0    \\
		\textbf{Autor}                & {Alejandro de Castro} \\
		\textbf{Requisitos asociados} & RF-2, RF-2.1, RF-2.2, RF-2.3 \\
		\textbf{Descripción}          & Permite al usuario crear una cuenta \\
		\textbf{Precondición}         & El usuario debe no haber iniciado sesión \\
		\textbf{Acciones}             &
		\begin{enumerate}
			\def\labelenumi{\arabic{enumi}.}
			\tightlist
			\item Rellenar los  campos de nombre de usuario, nombre completo, contraseña y confirmar contraseña.
			\item Pulsar en el botón de registrar.
		\end{enumerate}\\
		\textbf{Postcondición}        & Los datos son guardados en la base de datos y el usuario es creado, posteriormente es redirigido a la página de inicio de sesión. \\
		\textbf{Excepciones}          & En caso de no coincidir los campos de contraseñas o existir un usuario con ese nombre de usuario en la base de datos se notificará al usuario. \\
		\textbf{Importancia}          & Alta \\
		\bottomrule
	\end{tabularx}
	\caption{CU-1 Crear cuenta.}
\end{table}

% Caso de Uso 2 -> Iniciar sesión.
\begin{table}[p]
	\centering
	\begin{tabularx}{\linewidth}{ p{0.21\columnwidth} p{0.71\columnwidth} }
		\toprule
		\textbf{CU-2}    & \textbf{Iniciar sesión}\\
		\toprule
		\textbf{Versión}              & 1.0    \\
		\textbf{Autor}                & {Alejandro de Castro} \\
		\textbf{Requisitos asociados} & RF-1, RF-1.1, RF-1.2, RF-1.3 \\
		\textbf{Descripción}          & Permite al usuario iniciar sesión. \\
		\textbf{Precondición}         & \begin{itemize}
		    \item El usuario debe no haber iniciado sesión previamente. 
            \item El usuario debe tener una cuenta en el Sistema.
		\end{itemize} \\
		\textbf{Acciones}             &
		\begin{enumerate}
			\def\labelenumi{\arabic{enumi}.}
			\tightlist
			\item Rellenar los  campos de nombre de usuario y contraseña.
			\item Pulsar en el botón de iniciar sesión.
		\end{enumerate}\\
		\textbf{Postcondición}        & El usuario es redirigido a la página principal de la aplicación. \\
		\textbf{Excepciones}          & \begin{itemize}
		    \item Si el nombre de usuario no se encuentra en la base de datos se le notificará al usuario con un mensaje de error. 
            \item Si la contraseña es incorrecta se le notificará al usuario con un mensaje de error. 
		\end{itemize} \\
		\textbf{Importancia}          & Alta \\
		\bottomrule
	\end{tabularx}
	\caption{CU-2 Iniciar sesión.}
\end{table}

% Caso de Uso 3 -> Visualizar conjunto.
\begin{table}[p]
	\centering
	\begin{tabularx}{\linewidth}{ p{0.21\columnwidth} p{0.71\columnwidth} }
		\toprule
		\textbf{CU-3}    & \textbf{Visualizar conjunto}\\
		\toprule
		\textbf{Versión}              & 1.0    \\
		\textbf{Autor}                & {Alejandro de Castro} \\
		\textbf{Requisitos asociados} & RF-8 \\
		\textbf{Descripción}          & Permite al usuario visualizar un conjunto de datos. \\
		\textbf{Precondición}         & El usuario debe haber iniciado sesión. \\
		\textbf{Acciones}             &
		\begin{enumerate}
			\def\labelenumi{\arabic{enumi}.}
			\tightlist
			\item Seleccionar un conjunto de datos en la pantalla principal o en la página de conjuntos.
			\item Pulsar en el botón de consultar.
		\end{enumerate}\\
		\textbf{Postcondición}        & El usuario es redirigido a la página de visualización de conjuntos. \\
		\textbf{Excepciones}          & Si el conjunto de datos no está accesible se le redirigirá a la ventana de gestión de errores. \\
		\textbf{Importancia}          & Alta \\
		\bottomrule
	\end{tabularx}
	\caption{CU-3 Visualizar conjunto.}
\end{table}

% Caso de Uso 3.1 -> Personalizar conjunto.
\begin{table}[p]
	\centering
	\begin{tabularx}{\linewidth}{ p{0.21\columnwidth} p{0.71\columnwidth} }
		\toprule
		\textbf{CU-3.1}    & \textbf{Personalizar conjunto}\\
		\toprule
		\textbf{Versión}              & 1.0    \\
		\textbf{Autor}                & {Alejandro de Castro} \\
		\textbf{Requisitos asociados} & RF-8, RF-8.1, RF-8.2 \\
		\textbf{Descripción}          & Permite al usuario personalizar la vista que tiene del conjunto de datos. \\
		\textbf{Precondición}         & El usuario debe haber iniciado sesión y estar en la pantalla de visualizar conjunto. \\
		\textbf{Acciones}             &
		\begin{enumerate}
			\def\labelenumi{\arabic{enumi}.}
			\tightlist
			\item Seleccionar los parámetros deseados para mostrar en el \textit{tooltip} del mapa.
			\item Seleccionar en la tabla el parámetro por el que ordenar los conjuntos de datos.
		\end{enumerate}\\
		\textbf{Postcondición}        & El conjunto de datos se muestra de la forma indicada por el usuario. \\
		\textbf{Excepciones}          & Si el conjunto de datos deja de estar accesible se le redirigirá a la ventana de gestión de errores. \\
		\textbf{Importancia}          & Media \\
		\bottomrule
	\end{tabularx}
	\caption{CU-3.1 Personalizar conjunto.}
\end{table}

% Caso de Uso 4 -> Añadir conjunto.
\begin{table}[p]
	\centering
	\begin{tabularx}{\linewidth}{ p{0.21\columnwidth} p{0.71\columnwidth} }
		\toprule
		\textbf{CU-4}    & \textbf{Añadir conjunto}\\
		\toprule
		\textbf{Versión}              & 1.0    \\
		\textbf{Autor}                & {Alejandro de Castro} \\
		\textbf{Requisitos asociados} & RF-4, RF-4.1 \\
		\textbf{Descripción}          & Permite al usuario añadir un conjunto de datos a la aplicación. \\
		\textbf{Precondición}         & El usuario debe haber iniciado sesión y estar en la pantalla de conjuntos. \\
		\textbf{Acciones}             &
		\begin{enumerate}
			\def\labelenumi{\arabic{enumi}.}
			\tightlist
			\item Rellenar los campos de ciudad, conjunto, enlace, formato y opcionalmente los campos de periodicidad y unidades.
			\item Pulsar el botón de guardar.
		\end{enumerate}\\
		\textbf{Postcondición}        & El conjunto será añadido en la base de datos con los datos introducidos por el usuario. \\
		\textbf{Excepciones}          & Si alguno de los campos obligatorios no está completo, el botón de guardar se encontrará deshabilitado. \\
		\textbf{Importancia}          & Media \\
		\bottomrule
	\end{tabularx}
	\caption{CU-4 Añadir conjunto.}
\end{table}

% Caso de Uso 5 -> Gestionar registros.
\begin{table}[p]
	\centering
	\begin{tabularx}{\linewidth}{ p{0.21\columnwidth} p{0.71\columnwidth} }
		\toprule
		\textbf{CU-5}    & \textbf{Gestionar registros}\\
		\toprule
		\textbf{Versión}              & 1.0    \\
		\textbf{Autor}                & {Alejandro de Castro} \\
		\textbf{Requisitos asociados} & RF-5 \\
		\textbf{Descripción}          & Permite al usuario visualizar una lista de los registros de conjuntos de datos que está guardando el Sistema para él. \\
		\textbf{Precondición}         & El usuario debe haber iniciado sesión. \\
		\textbf{Acciones}             & Pulsar en el botón de registros de la barra de navegación.\\
		\textbf{Postcondición}        & Se mostrará una lista con los registros que el Sistema está realizando para el usuario con las opciones de descargar y eliminar. \\
		\textbf{Excepciones}          & Si no hay ningún registro, se mostrará la lista vacía y la opción de ñadir registros. \\
		\textbf{Importancia}          & Alta \\
		\bottomrule
	\end{tabularx}
	\caption{CU-5 Gestionar registros.}
\end{table}

% Caso de Uso 5.1 -> Añadir registro.
\begin{table}[p]
	\centering
	\begin{tabularx}{\linewidth}{ p{0.21\columnwidth} p{0.71\columnwidth} }
		\toprule
		\textbf{CU-5.1}    & \textbf{Añadir registro}\\
		\toprule
		\textbf{Versión}              & 1.0    \\
		\textbf{Autor}                & {Alejandro de Castro} \\
		\textbf{Requisitos asociados} & RF-5, RF-5.1 \\
		\textbf{Descripción}          & Permite al usuario comenzar el registro de un conjunto de datos determinado con una periodicidad determinada por él. \\
		\textbf{Precondición}         & El usuario debe haber iniciado sesión y debe haber conjuntos de datos almacenados en la base de datos. \\
        \textbf{Acciones}             &
		\begin{enumerate}
			\def\labelenumi{\arabic{enumi}.}
			\tightlist
			\item Seleccionar un conjunto, rellenar la periodicidad y unidades.
			\item Pulsar el botón de guardar.
		\end{enumerate}\\ 
		\textbf{Postcondición}        & Se añadirá la base de datos los datos seleccionados por el usuario y el Sistema comenzará al registro del conjunto de datos con la periodicidad determinada por el usuario. \\
		\textbf{Excepciones}          & Si la periodicidad es menor que la periodicidad de actualización de dicho conjunto se mostrará un mensaje al usuario indicando que debe ser una periodicidad mayor y se desahbilitará el botón de guardar. \\
		\textbf{Importancia}          & Alta \\
		\bottomrule
	\end{tabularx}
	\caption{CU-5.1 Añadir registro.}
\end{table}

% Caso de Uso 5.2 -> Descargar registro.
\begin{table}[p]
	\centering
	\begin{tabularx}{\linewidth}{ p{0.21\columnwidth} p{0.71\columnwidth} }
		\toprule
		\textbf{CU-5.2}    & \textbf{Descargar registro}\\
		\toprule
		\textbf{Versión}              & 1.0    \\
		\textbf{Autor}                & {Alejandro de Castro} \\
		\textbf{Requisitos asociados} & RF-5.2 \\
		\textbf{Descripción}          & Permite al usuario descargar el registro seleccionado. \\
		\textbf{Precondición}         & El usuario debe haber iniciado sesión y debe tener al menos un registro. \\
        \textbf{Acciones}             &
		\begin{enumerate}
			\def\labelenumi{\arabic{enumi}.}
			\tightlist
			\item Seleccionar un registro.
			\item Pulsar el botón de descargar.
		\end{enumerate}\\ 
		\textbf{Postcondición}        & Se descargará en formato \textit{zip} el registro indicado por el usuario. \\
		\textbf{Excepciones}          & Si ocurre algún error durante la descarga el usuario será redirigido a la pantalla de gestión de errores. \\
		\textbf{Importancia}          & Alta \\
		\bottomrule
	\end{tabularx}
	\caption{CU-5.2 Descargar registro.}
\end{table}

% Caso de Uso 5.3 -> Eliminar registro.
\begin{table}[p]
	\centering
	\begin{tabularx}{\linewidth}{ p{0.21\columnwidth} p{0.71\columnwidth} }
		\toprule
		\textbf{CU-5.3}    & \textbf{Eliminar registro}\\
		\toprule
		\textbf{Versión}              & 1.0    \\
		\textbf{Autor}                & {Alejandro de Castro} \\
		\textbf{Requisitos asociados} & RF-5.3 \\
		\textbf{Descripción}          & Permite al usuario dejar de registrar un conjunto de datos. \\
		\textbf{Precondición}         & El usuario debe haber iniciado sesión y debe tener al menos un registro. \\
        \textbf{Acciones}             &
		\begin{enumerate}
			\def\labelenumi{\arabic{enumi}.}
			\tightlist
			\item Seleccionar un registro.
			\item Pulsar el botón de eliminar.
		\end{enumerate}\\ 
		\textbf{Postcondición}        & Se terminará con el registro de ese conjunto para ese usuario. \\
		\textbf{Excepciones}          & Si ocurre algún error durante la eliminación de ese conjunto, el usuario será redirigido a la pantalla de gestión de errores. \\
		\textbf{Importancia}          & Alta \\
		\bottomrule
	\end{tabularx}
	\caption{CU-5.3 Eliminar registro.}
\end{table}

% Caso de Uso 6 -> Reportar error.
\begin{table}[p]
	\centering
	\begin{tabularx}{\linewidth}{ p{0.21\columnwidth} p{0.71\columnwidth} }
		\toprule
		\textbf{CU-6}    & \textbf{Reportar error}\\
		\toprule
		\textbf{Versión}              & 1.0    \\
		\textbf{Autor}                & {Alejandro de Castro} \\
		\textbf{Requisitos asociados} & RF-9 \\
		\textbf{Descripción}          & Permite al usuario dejar constancia de un error producido en la aplicación. \\
		\textbf{Precondición}         & El usuario debe haber iniciado sesión y encontrarse en la pantalla de gestión de errores tras haberse producido un error. \\
        \textbf{Acciones}             &
		\begin{enumerate}
			\def\labelenumi{\arabic{enumi}.}
			\tightlist
			\item Pulsar el botón reportar error.
		\end{enumerate}\\ 
		\textbf{Postcondición}        & Se guardará el error encontrado, la fecha y el usuario para su revisión y el usuario. \\
		\textbf{Excepciones}          & No hay excepciones en esta página al ser crítica no se permiten fallos. \\
		\textbf{Importancia}          & Alta \\
		\bottomrule
	\end{tabularx}
	\caption{CU-6 Reportar error.}
\end{table}

% Caso de Uso 7 -> Gestionar favoritos.
\begin{table}[p]
	\centering
	\begin{tabularx}{\linewidth}{ p{0.21\columnwidth} p{0.71\columnwidth} }
		\toprule
		\textbf{CU-7}    & \textbf{Gestionar favoritos}\\
		\toprule
		\textbf{Versión}              & 1.0    \\
		\textbf{Autor}                & {Alejandro de Castro} \\
		\textbf{Requisitos asociados} & RF-3, RF-3.1 \\
		\textbf{Descripción}          & Permite al usuario visualizar una lista de los conjuntos que tiene en favoritos. \\
		\textbf{Precondición}         & El usuario debe haber iniciado sesión. \\
        \textbf{Acciones}             &
		\begin{enumerate}
			\def\labelenumi{\arabic{enumi}.}
			\tightlist
			\item Pulsar en el botón \textit{home} de la barra de navegación.
		\end{enumerate}\\ 
		\textbf{Postcondición}        & Se mostrará una lista con los conjuntos establecidos como favoritos por el usuario. \\
		\textbf{Excepciones}          & En caso de no tener ningún conjunto marcado como favoritos se mostrará una lista vacía. \\
		\textbf{Importancia}          & Alta \\
		\bottomrule
	\end{tabularx}
	\caption{CU-7 Gestionar favoritos.}
\end{table}

% Caso de Uso 7.1 -> Marcar favorito.
\begin{table}[p]
	\centering
	\begin{tabularx}{\linewidth}{ p{0.21\columnwidth} p{0.71\columnwidth} }
		\toprule
		\textbf{CU-7.1}    & \textbf{Marcar favorito}\\
		\toprule
		\textbf{Versión}              & 1.0    \\
		\textbf{Autor}                & {Alejandro de Castro} \\
		\textbf{Requisitos asociados} & RF-3 \\
		\textbf{Descripción}          & Permite al usuario marcar como favorito un conjunto de datos. \\
		\textbf{Precondición}         & \begin{itemize}
		    \item El usuario debe haber iniciado sesión. 
            \item El usuario debe encontrarse en la pantalla de visualización de un conjunto. 
            \item El usuario no debe tener ese conjunto marcado como favorito previamente. 
		\end{itemize} \\
        \textbf{Acciones}             &
		\begin{enumerate}
			\def\labelenumi{\arabic{enumi}.}
			\tightlist
			\item Pulsar en el botón en forma de corazón negro.
		\end{enumerate} \\ 
		\textbf{Postcondición}        & Se añadirá el conjunto a la lista de favoritos de ese usuario en la base de datos y el botón en forma de corazón negro quedará coloreado de rojo. \\
		\textbf{Excepciones}          & En caso de no poderse añadir a favoritos se direccionara al usuario a la pantalla de gestión de errores. \\
		\textbf{Importancia}          & Media \\
		\bottomrule
	\end{tabularx}
	\caption{CU-7.1 Marcar favorito.}
\end{table}

% Caso de Uso 7.2 -> Desmarcar favorito.
\begin{table}[p]
	\centering
	\begin{tabularx}{\linewidth}{ p{0.21\columnwidth} p{0.71\columnwidth} }
		\toprule
		\textbf{CU-7.2}    & \textbf{Desmarcar favorito}\\
		\toprule
		\textbf{Versión}              & 1.0    \\
		\textbf{Autor}                & {Alejandro de Castro} \\
		\textbf{Requisitos asociados} & RF-3 \\
		\textbf{Descripción}          & Permite al usuario desmarcar como favorito un conjunto de datos. \\
		\textbf{Precondición}         & \begin{itemize}
		    \item El usuario debe haber iniciado sesión. 
            \item El usuario debe encontrarse en la pantalla de visualización de un conjunto. 
            \item El usuario debe tener ese conjunto marcado como favorito previamente. 
		\end{itemize} \\
        \textbf{Acciones}             &
		\begin{enumerate}
			\def\labelenumi{\arabic{enumi}.}
			\tightlist
			\item Pulsar en el botón en forma de corazón rojo.
		\end{enumerate}\\ 
		\textbf{Postcondición}        & Se eliminará el conjunto a la lista de favoritos de ese usuario en la base de datos y el botón en forma de corazón cambiará de color a negro. \\
		\textbf{Excepciones}          & En caso de no poderse eliminar de favoritos se direccionará al usuario a la pantalla de gestión de errores. \\
		\textbf{Importancia}          & Media \\
		\bottomrule
	\end{tabularx}
	\caption{CU-7.2 Desmarcar favorito.}
\end{table}

% Caso de Uso 8 -> Visualizar ayuda.
\begin{table}[p]
	\centering
	\begin{tabularx}{\linewidth}{ p{0.21\columnwidth} p{0.71\columnwidth} }
		\toprule
		\textbf{CU-8}    & \textbf{Visualizar ayuda}\\
		\toprule
		\textbf{Versión}              & 1.0    \\
		\textbf{Autor}                & {Alejandro de Castro} \\
		\textbf{Requisitos asociados} & RF-10 \\
		\textbf{Descripción}          & Permite al usuario visualizar una pantalla con información sobre el funcionamiento de la aplicación y el contacto de los administradores principales. \\
		\textbf{Precondición}         & El usuario debe haber iniciado sesión. \\
        \textbf{Acciones}             &
		\begin{enumerate}
			\def\labelenumi{\arabic{enumi}.}
			\tightlist
			\item Pulsar en el botón ayuda de la barra de navegación.
		\end{enumerate}\\ 
		\textbf{Postcondición}        & Se mostrará una pantalla con información sobre el funcionamiento de la aplicación y el contacto de los administradores principales. \\
		\textbf{Excepciones}          & Este apartado es crítico, por lo que no hay excepciones permitidas. \\
		\textbf{Importancia}          & Alta \\
		\bottomrule
	\end{tabularx}
	\caption{CU-8 Visualizar ayuda.}
\end{table}

% Caso de Uso 9 -> Gestionar usuarios.
\begin{table}[p]
	\centering
	\begin{tabularx}{\linewidth}{ p{0.21\columnwidth} p{0.71\columnwidth} }
		\toprule
		\textbf{CU-9}    & \textbf{Gestionar usuarios}\\
		\toprule
		\textbf{Versión}              & 1.0    \\
		\textbf{Autor}                & {Alejandro de Castro} \\
		\textbf{Requisitos asociados} & RF-6 \\
		\textbf{Descripción}          & Permite al administrador visualizar una lista de los usuarios registrados en el Sistema. \\
		\textbf{Precondición}         & El administrador debe haber iniciado sesión. \\
        \textbf{Acciones}             &
		\begin{enumerate}
			\def\labelenumi{\arabic{enumi}.}
			\tightlist
			\item Pulsar en el botón gestión de usuarios en la pantalla principal del administrador.
		\end{enumerate}\\ 
		\textbf{Postcondición}        & Se mostrará una lista con los usuarios registrados en el Sistema. \\
		\textbf{Excepciones}          & En caso de que el rol del administrador haya cambiado durante el proceso se le dirigirá a la pantalla de error notificándole que no tiene permisos para modificar usuarios. \\
		\textbf{Importancia}          & Alta \\
		\bottomrule
	\end{tabularx}
	\caption{CU-9 Gestionar usuarios.}
\end{table}

% Caso de Uso 9.1 -> Modificar usuario.
\begin{table}[p]
	\centering
	\begin{tabularx}{\linewidth}{ p{0.21\columnwidth} p{0.71\columnwidth} }
		\toprule
		\textbf{CU-9.1}    & \textbf{Modificar usuario}\\
		\toprule
		\textbf{Versión}              & 1.0    \\
		\textbf{Autor}                & {Alejandro de Castro} \\
		\textbf{Requisitos asociados} & RF-6, RF-6.1 \\
		\textbf{Descripción}          & Permite al administrador modificar el rol de los usuarios. \\
		\textbf{Precondición}         & El administrador debe haber iniciado sesión y estar en la pantalla de gestión de usuarios. \\
        \textbf{Acciones}             &
		\begin{enumerate}
			\def\labelenumi{\arabic{enumi}.}
			\tightlist
            \item Seleccionar el usuario de la lista de usuarios.
			\item Seleccionar el rol al que se desea cambiar.
		\end{enumerate}\\ 
		\textbf{Postcondición}        & Se mostrará un mensaje con la confirmación del cambio de rol. \\
		\textbf{Excepciones}          & En caso de no haber podido realizar la modificación se notificará al administrador. \\
		\textbf{Importancia}          & Media \\
		\bottomrule
	\end{tabularx}
	\caption{CU-9.1 Modificar usuario.}
\end{table}

% Caso de Uso 9.2 -> Eliminar usuario.
\begin{table}[p]
	\centering
	\begin{tabularx}{\linewidth}{ p{0.21\columnwidth} p{0.71\columnwidth} }
		\toprule
		\textbf{CU-9.2}    & \textbf{Eliminar usuario}\\
		\toprule
		\textbf{Versión}              & 1.0    \\
		\textbf{Autor}                & {Alejandro de Castro} \\
		\textbf{Requisitos asociados} & RF-6, RF-6.2 \\
		\textbf{Descripción}          & Permite al administrador eliminar usuarios. \\
		\textbf{Precondición}         & El administrador debe haber iniciado sesión y estar en la pantalla de gestión de usuarios. \\
        \textbf{Acciones}             &
		\begin{enumerate}
			\def\labelenumi{\arabic{enumi}.}
			\tightlist
            \item Seleccionar el usuario de la lista de usuarios.
			\item Pulsar el botón eliminar.
		\end{enumerate}\\ 
		\textbf{Postcondición}        & El usuario desaparecerá de la lista de usuarios. \\
		\textbf{Excepciones}          & En caso de no haber podido realizar la eliminación se direccionará al administrador a la pantalla de errores y se le informará del error. \\
		\textbf{Importancia}          & Alta \\
		\bottomrule
	\end{tabularx}
	\caption{CU-9.2 Eliminar usuario.}
\end{table}

% Caso de Uso 10 -> Eliminar conjunto.
\begin{table}[p]
	\centering
	\begin{tabularx}{\linewidth}{ p{0.21\columnwidth} p{0.71\columnwidth} }
		\toprule
		\textbf{CU-10}    & \textbf{Eliminar conjunto}\\
		\toprule
		\textbf{Versión}              & 1.0    \\
		\textbf{Autor}                & {Alejandro de Castro} \\
		\textbf{Requisitos asociados} & RF-6, RF-6.2 \\
		\textbf{Descripción}          & Permite al administrador eliminar un conjunto. \\
		\textbf{Precondición}         & El administrador debe haber iniciado sesión y estar en la pantalla de conjuntos. \\
        \textbf{Acciones}             &
		\begin{enumerate}
			\def\labelenumi{\arabic{enumi}.}
			\tightlist
            \item Seleccionar un conjunto de la lista de conjuntos.
			\item Pulsar el botón eliminar.
		\end{enumerate}\\ 
		\textbf{Postcondición}        & El conjunto desaparecerá de la lista de conjuntos. \\
		\textbf{Excepciones}          & En caso de no haber podido realizar la eliminación del conjunto se direccionará al administrador a la pantalla de errores y se le informará del error. \\
		\textbf{Importancia}          & Alta \\
		\bottomrule
	\end{tabularx}
	\caption{CU-10 Eliminar conjunto.}
\end{table}

% Caso de Uso 11 -> Gestionar traducciones.
\begin{table}[p]
	\centering
	\begin{tabularx}{\linewidth}{ p{0.21\columnwidth} p{0.71\columnwidth} }
		\toprule
		\textbf{CU-11}    & \textbf{Gestionar traducciones}\\
		\toprule
		\textbf{Versión}              & 1.0    \\
		\textbf{Autor}                & {Alejandro de Castro} \\
		\textbf{Requisitos asociados} & RF-7, RF-7.1 \\
		\textbf{Descripción}          & Permite al administrador visualizar una lista de las traducciones guardadas en el Sistema. \\
		\textbf{Precondición}         & El administrador debe haber iniciado sesión. \\
        \textbf{Acciones}             &
		\begin{enumerate}
			\def\labelenumi{\arabic{enumi}.}
			\tightlist
			\item Pulsar el botón gestionar traducciones en la pantalla principal del administrador.
		\end{enumerate}\\ 
		\textbf{Postcondición}        & Se mostrará al administrador una lista de las traducciones guardadas en el Sistema. \\
		\textbf{Excepciones}          & En caso de no haber ninguna traducción se mostrará al administrador una lista vacía. \\
		\textbf{Importancia}          & Alta \\
		\bottomrule
	\end{tabularx}
	\caption{CU-11 Gestionar traducciones.}
\end{table}

% Caso de Uso 11.1 -> Añadir traducción.
\begin{table}[p]
	\centering
	\begin{tabularx}{\linewidth}{ p{0.21\columnwidth} p{0.71\columnwidth} }
		\toprule
		\textbf{CU-11.1}    & \textbf{Añadir traducción}\\
		\toprule
		\textbf{Versión}              & 1.0    \\
		\textbf{Autor}                & {Alejandro de Castro} \\
		\textbf{Requisitos asociados} & RF-7, RF-7.1 \\
		\textbf{Descripción}          & Permite al administrador añadir una traducción. \\
		\textbf{Precondición}         & El administrador debe haber iniciado sesión y estar en la pantalla de gestionar traducciones. \\
        \textbf{Acciones}             &
		\begin{enumerate}
			\def\labelenumi{\arabic{enumi}.}
			\tightlist
            \item Rellenar los campos de palabra original y palabra traducida.
			\item Pulsar el botón añadir.
		\end{enumerate}\\ 
		\textbf{Postcondición}        & Se añadirá la traducción al Sistema y aparecerá en la lista de traducciones. \\
		\textbf{Excepciones}          & En caso de no haber podido guardar la traducción se direccionará al administrador a la pantalla de errores y se le informará del error. \\
		\textbf{Importancia}          & Media \\
		\bottomrule
	\end{tabularx}
	\caption{CU-11.1 Añadir traducción.}
\end{table}

% Caso de Uso 11.2 -> Eliminar traducción.
\begin{table}[p]
	\centering
	\begin{tabularx}{\linewidth}{ p{0.21\columnwidth} p{0.71\columnwidth} }
		\toprule
		\textbf{CU-11.2}    & \textbf{Eliminar traducción}\\
		\toprule
		\textbf{Versión}              & 1.0    \\
		\textbf{Autor}                & {Alejandro de Castro} \\
		\textbf{Requisitos asociados} & RF-7, RF-7.1 \\
		\textbf{Descripción}          & Permite al administrador eliminar una traducción. \\
		\textbf{Precondición}         & El administrador debe haber iniciado sesión y estar en la pantalla de gestionar traducciones. \\
        \textbf{Acciones}             &
		\begin{enumerate}
			\def\labelenumi{\arabic{enumi}.}
			\tightlist
            \item Seleccionar una traducción de la lista de traducciones.
			\item Pulsar el botón eliminar.
		\end{enumerate}\\ 
		\textbf{Postcondición}        & La traducción desaparecerá de la lista de traducciones. \\
		\textbf{Excepciones}          & En caso de no haber podido realizar la eliminación de la traducción se direccionará al administrador a la pantalla de errores y se le informará del error. \\
		\textbf{Importancia}          & Media \\
		\bottomrule
	\end{tabularx}
	\caption{CU-11.2 Eliminar traducción.}
\end{table}