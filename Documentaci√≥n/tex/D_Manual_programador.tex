\apendice{Documentación técnica de programación}

\section{Introducción}
En esta sección se recoge la información más relevante relacionada el código de la aplicación, así como las herramientas clave para su desarrollo, compilación y ejecución. También se incluyen distintas pruebas que se han realizado para asegurar el correcto funcionamiento del mismo.
Esta sección tiene la siguiente estructura:
\begin{itemize}
    \item \textbf{Estructura de directorios:} contiene la estructura del repositorio donde se encuentra el código.
    \item \textbf{Manual del programador:} se detalla el software empleado para el desarrollo de la aplicación.
    \item \textbf{Compilación, instalación y ejecución del proyecto:} contiene una guía para la compilación, instalación y ejecución del proyecto
    \item \textbf{Pruebas del sistema:} se detallan distintas pruebas que se han hecho en la aplicación.
\end{itemize}

\section{Estructura de directorios}

El repositorio de \textit{GitHub}\footnote{\url{https://github.com/AlejandroDeCastro/TFG}} donde se en encuentra el proyecto, tiene la siguiente estructura:
\begin{itemize}
    \item \textbf{CODE:} Directorio donde se encuentra el código de la aplicación. Tiene la siguiente estructura:
    \begin{itemize}
        \item \textbf{assets:} Contiene los estilos \textit{css} de \textit{Dash}:
        \begin{itemize}
            \item \textbf{stylesheet.css:} Fichero \textit{css} que contiene el diseño de las vistas hechas con \textit{Dash}.
            \item \textbf{s1.css:} Fichero \textit{css} que contiene el diseño de las gráficas de las vistas hechas con \textit{Dash}.
        \end{itemize}
        \item \textbf{models:} Contiene los modelos de las entidades que ejecutan el programa y las acciones relacionadas con ellas:
        \begin{itemize}
            \item \textbf{entidades:} Directorio que contiene las distintas entidades del sistema:
            \begin{itemize}
                \item \textbf{Usuario.py:} Contiene la clase "Usuario". 
            \end{itemize}
            \item \textbf{ModeloUsuario.py:} Contiene la clase ``ModeloUsuario'' en \textit{Python}, la cual alberga todas acciones relacionadas con el usuario.
        \end{itemize}
        \item \textbf{static:} Directorio donde se encuentran los ficheros que dan soporte al \textit{front-end} de la aplicación y algunas funcionalidades en las propias pantallas de la aplicación:
        \begin{itemize}
            \item \textbf{css:} Directorio donde se encuentran todos los ficheros \textit{css} de la aplicación, encargados de aportar la estética de las distintas vistas.
            \item \textbf{img:} Directorio con las distintas imágenes de la aplicación.
            \item \textbf{js:} Directorio que contiene todos los ficheros \textit{js} de la aplicación, son los encargados de añadir algunas funcionalidades a las distintas vistas.
        \end{itemize}
        \item \textbf{templates:} Directorio que contiene los ficheros \textit{html} de la aplicación, estos ficheros aportan las vistas de la aplicación.
        \item \textbf{app.py:} \textit{Script} principal de la aplicación, programado en \textit{Python}, contiene todas las rutas de la aplicación.
        \item \textbf{database.py:} Fichero de \textit{Python} que contiene el conector con la base de datos.
        \item \textbf{gestor.py:} Fichero de \textit{Python} que contiene todas las interacciones con la base de datos, menos las que están relacionadas con el usuario. También es el \textit{script} encargado de la gestión de registros.
    \end{itemize}
    \item \textbf{Documentación:} Directorio donde se encuentra toda la documentación del proyecto, organizada en documentos \LaTeX{}
    \item \textbf{Registros:} Directorio donde se almacenan los registros solicitados por cada usuario.
    \item \textbf{Server:} Directorio donde se encuentra el \textit{OCB} y todos los ficheros relacionados con él:
    \begin{itemize}
        \item \textbf{Server/Ficheros:} Directorio donde se encuentra un histórico de los ficheros \textit{JSON} del simulador.
        \item \textbf{menu.bat:} \textit{Script} que proporciona al usuario un menú desde el cual gestionar el \textit{OCB} en \textit{Windwos}.
        \item \textbf{menu.sh:} \textit{Script} que proporciona al usuario un menú desde el cual gestionar el \textit{OCB} en \textit{Linux}.
        \item \textbf{generadorDatos.py:} Simulador que proporciona datos al \textit{OCB}.
        \item \textbf{limpiador.py:} \textit{Script} que eliminar todas las entidades almacenadas en el \textit{OCB}.
        \item \textbf{simulador.txt:} Fichero de texto plano que almacena el \textit{id} del proceso que está ejecutando el simulador para detenerlo cuando se ejecuta en segundo plano.
        \item \textbf{transformadorDatos.py:} \textit{Script}que transforma ficheros \textit{JSON} al modelo \textit{NGSI}, actualmente en desuso.
    \end{itemize}
    \item \textbf{db:} Directorio que contiene la base de datos de la aplicación.
    \begin{itemize}
        \item \textbf{tfg.sql:} Base de datos de la aplicación.
    \end{itemize}
    \item \textbf{Dockerfile:} Fichero de texto plano que contiene una serie de instrucciones necesarias para crear la imagen \textit{Docker} que contiene la aplicación, para su distribución o despliegue en un servidor real.
    \item \textbf{LICENSE:} Licencia para la  distribución del código de la aplicación.
    \item \textbf{Readme.md:} Fichero que proporciona información útil sobre el proyecto, como la licencia, enlaces de interés, descripción, etc.
    \item \textbf{docker-compose.yml:} Fichero de configuración para la creación del \textit{Docker} mencionado anteriormente.
    \item \textbf{error\_log.txt:} Fichero de texto plano donde se almacenan los reportes de los usuarios cuando ha ocurrido un error.
    \item \textbf{requirements.txt:} Librerías esenciales y sus versiones para la ejecución de la aplicación.
\end{itemize}

Se han omitido los directorios \textbf{\_\_pycache\_\_} que contienen \textit{scripts} ya compilados de \textit{Python}, se genera automáticamente. No son relevantes.


\section{Manual del programador}

Esta sección tiene como objetivo servir de guía a futuros programadores que deseen continuar con el desarrollo de la aplicación. Es por ello que, a continuación voy a detallar como descargar el código e instalar el software necesario para su uso.

\subsection{Entorno de desarrollo}

El entorno de desarrollo recomendado es \textit{Visual Studio 2022}\footnote{\url{https://visualstudio.microsoft.com/es/}}, ya que ha sido la empleada para el desarrollo de la aplicación y e incorpora herramientas para el desarrollo de código en diversos lenguajes usados en la aplicación (\textit{Python}, \textit{HTML5}, \textit{CSS3}, \textit{JavaScript}, etc). 

\textit{Visual Studio 2022} es un \textit{IDE} completo, extensible y gratuito, con conectividad con \textit{GitHub}\footnote{Repositorio del código fuente: \url{https://github.com/AlejandroDeCastro/TFG}}, lo cual permite clonar el repositorio con el código fuente y comenzar a modificarlo de una manera muy sencilla.

La aplicación en sí, está desarrollada en \textit{Python} y necesita determinadas librerías para su correcto funcionamiento. Es por ello, por lo que existe un fichero llamado ``requirements.txt'' en el repositorio, el cual contiene las librerías y las versiones necesarias. Para instalar todas las dependencias, basta con ejecutar el siguiente comando en el directorio raíz del proyecto: \texttt{\$ pip3 install -r requirements.txt}.

\subsection{Base de datos}

La base de datos empleada es una base de datos relacional, la cual se ha gestionado con \textit{MySQL}, más en concreto con \textit{XAMPP}\footnote{\url{https://www.apachefriends.org/es/index.html}}, una herramienta que permite trabajar en local con la base datos y gestionarla mediante una interfaz gráfica. Al ser un software libre puede ser usado por cualquier usuario de manera completamente gratuita.

\subsection{Dockers}

Por último, para la instalación del \textit{Orion Context Broker} es necesario tener instalado \textit{Docker} en el sistema operativo, ya que, precisa de dos \textit{dockers}: \textit{fiware-orion} y \textit{mongoDB}.
La instalación y gestión de ambos \textit{dockers} se puede hacer cómodamente ejecutando el \textit{spript} ``menu.bat'' si se desea instalar desde \textit{Windows} o ``menu.sh'' si es desde un sistema \textit{Unix}.

\section{Compilación, instalación y ejecución del proyecto}

En este aparatado se detalla como compilar, instalar y ejecutar el proyecto en cinco sencillos pasos:
\begin{enumerate}
    \item \textbf{Clonar el repositorio:} Lo primero de todo es clonar el repositorio de \textit{GitHub} mencionado en el apartado anterior.
    \item \textbf{Acceder a la carpeta raíz del directorio clonado:} Entrar con el comando: \texttt{cd TFG}.
    \item \textbf{Generar la imagen de \textit{Docker}:} Con el comando \texttt{docker build -t SmartCitiesApp}.
    \item \textbf{Ejecutar el \textit{Docker}:} Usando el comando \texttt{docker-compose up}.
    \item \textbf{Acceder a la aplicación:} Mediante el enlace \url{http://127.0.0.1:5000/} .
\end{enumerate}

\section{Pruebas del sistema}

Se han elaborado algunas pruebas del sistema para verificar que no haya errores.

%CP1
\begin{longtable}{>{\hspace{0pt}}m{0.25\linewidth}>{\hspace{0pt}}m{0.764\linewidth}}
\label{CP1}
\caption{CP-1 Crear cuenta con usuario existente}\\ 
\hline
\rowcolor{gray!20} \multicolumn{1}{|>{\hspace{0pt}}m{0.25\linewidth}|}{\textbf{CP-1}} & \multicolumn{1}{>{\hspace{0pt}}m{0.764\linewidth}|}{Crear cuenta con usuario existente} \endfirsthead 
\hline
\textbf{Importancia} & Alta \\
\rowcolor{gray!20} \textbf{Descripción} & No se permite la creación de una cuenta con un nombre de usuario ya en uso. \\
\textbf{Precondición} & Introducir un nombre de usuario ya existente en la aplicación a la hora de crear una cuenta nueva. \\
\rowcolor{gray!20} \textbf{Acciones} & Se comprueba si el nombre de usuario existe en la base de datos. \\
\textbf{Postcondición} & Se notifica al usuario que debe escoger otro nombre de usuario. \\
\hline
\end{longtable}

%CP2
\begin{longtable}{>{\hspace{0pt}}m{0.25\linewidth}>{\hspace{0pt}}m{0.764\linewidth}}
\label{CP2}
\caption{CP-2 Contraseñas no coinciden al crear una cuenta}\\ 
\hline
\rowcolor{gray!20} \multicolumn{1}{|>{\hspace{0pt}}m{0.25\linewidth}|}{\textbf{CP-2}} & \multicolumn{1}{>{\hspace{0pt}}m{0.764\linewidth}|}{Contraseñas no coinciden al crear una cuenta} \endfirsthead 
\hline
\textbf{Importancia} & Alta \\
\rowcolor{gray!20} \textbf{Descripción} & No permitir al usuario escribir una contraseña distinta en el campo de confirmación de contraseña respecto al campo de contraseña. \\
\textbf{Precondición} & Introducir dos contraseñas distintas en el formulario de crear una cuenta nueva. \\
\rowcolor{gray!20} \textbf{Acciones} & Se comprueba si ambas contraseñas coinciden. \\
\textbf{Postcondición} & Se notifica al usuario que las contraseñas no coinciden. \\
\hline
\end{longtable}

%CP3
\begin{longtable}{>{\hspace{0pt}}m{0.25\linewidth}>{\hspace{0pt}}m{0.764\linewidth}}
\label{CP3}
\caption{CP-3 Usuario inexistente al iniciar sesión}\\ 
\hline
\rowcolor{gray!20} \multicolumn{1}{|>{\hspace{0pt}}m{0.25\linewidth}|}{\textbf{CP-3}} & \multicolumn{1}{>{\hspace{0pt}}m{0.764\linewidth}|}{Usuario inexistente al iniciar sesión} \endfirsthead 
\hline
\textbf{Importancia} & Media \\
\rowcolor{gray!20} \textbf{Descripción} & Avisar al usuario que el usuario no existe en caso de que intente iniciar sesión con un usuario no registrado. \\
\textbf{Precondición} & Introducir un nombre de usuario que no está registrado en la aplicación al iniciar sesión. \\
\rowcolor{gray!20} \textbf{Acciones} & Se comprueba si el usuario existe. \\
\textbf{Postcondición} & Se notifica al usuario que el usuario introducido no existe. \\
\hline
\end{longtable}

%CP4
\begin{longtable}{>{\hspace{0pt}}m{0.25\linewidth}>{\hspace{0pt}}m{0.764\linewidth}}
\label{CP4}
\caption{CP-4 Contraseña incorrecta}\\ 
\hline
\rowcolor{gray!20} \multicolumn{1}{|>{\hspace{0pt}}m{0.25\linewidth}|}{\textbf{CP-4}} & \multicolumn{1}{>{\hspace{0pt}}m{0.764\linewidth}|}{Contraseña incorrecta} \endfirsthead 
\hline
\textbf{Importancia} & Alta \\
\rowcolor{gray!20} \textbf{Descripción} & Avisar al usuario que la contraseña introducida no es correcta. \\
\textbf{Precondición} & Introducir una contraseña que incorrecta al intentar iniciar sesión. \\
\rowcolor{gray!20} \textbf{Acciones} & Se comprueba si la contraseña es correcta. \\
\textbf{Postcondición} & Se notifica al usuario que la contraseña introducida es incorrecta. \\
\hline
\end{longtable}

%CP5
\begin{longtable}{>{\hspace{0pt}}m{0.25\linewidth}>{\hspace{0pt}}m{0.764\linewidth}}
\label{CP5}
\caption{CP-5 Conjunto de datos borrado}\\ 
\hline
\rowcolor{gray!20} \multicolumn{1}{|>{\hspace{0pt}}m{0.25\linewidth}|}{\textbf{CP-5}} & \multicolumn{1}{>{\hspace{0pt}}m{0.764\linewidth}|}{Conjunto de datos borrado} \endfirsthead 
\hline
\textbf{Importancia} & Alta \\
\rowcolor{gray!20} \textbf{Descripción} & Cuando un conjunto de datos es borrado, se elimina de los favoritos de los usuarios y se detienen los registros que accedan a ese conjunto. \\
\textbf{Precondición} & Eliminar un conjunto de datos de la aplicación. \\
\rowcolor{gray!20} \textbf{Acciones} & Se elimina de favoritos en los usuarios que lo tuvieran escogido y se finalizan los registros de dicho conjunto. \\
\textbf{Postcondición} & La aplicación sigue funcionando sin ese conjunto de datos. \\
\hline
\end{longtable}

%CP6
\begin{longtable}{>{\hspace{0pt}}m{0.25\linewidth}>{\hspace{0pt}}m{0.764\linewidth}}
\label{CP6}
\caption{CP-6 Conjunto de datos inaccesible}\\ 
\hline
\rowcolor{gray!20} \multicolumn{1}{|>{\hspace{0pt}}m{0.25\linewidth}|}{\textbf{CP-6}} & \multicolumn{1}{>{\hspace{0pt}}m{0.764\linewidth}|}{Conjunto de datos inaccesible} \endfirsthead 
\hline
\textbf{Importancia} & Alta \\
\rowcolor{gray!20} \textbf{Descripción} & Cuando se intenta acceder a un conjunto de datos el cual no está disponible en ese momento, se notifica al usuario y se le da la posibilidad de reportar el error. \\
\textbf{Precondición} & intentar acceder a un conjunto de datos inaccesible. \\
\rowcolor{gray!20} \textbf{Acciones} & Comprobar el acceso al conjunto de datos. \\
\textbf{Postcondición} & Se notifica al usuario y se le redirige a la ventana de errores. \\
\hline
\end{longtable}

%CP7
\begin{longtable}{>{\hspace{0pt}}m{0.25\linewidth}>{\hspace{0pt}}m{0.764\linewidth}}
\label{CP7}
\caption{CP-7 Acceso a una funcionalidad restringida}\\ 
\hline
\rowcolor{gray!20} \multicolumn{1}{|>{\hspace{0pt}}m{0.25\linewidth}|}{\textbf{CP-7}} & \multicolumn{1}{>{\hspace{0pt}}m{0.764\linewidth}|}{Acceso a una funcionalidad restringida} \endfirsthead 
\hline
\textbf{Importancia} & Alta \\
\rowcolor{gray!20} \textbf{Descripción} & Cuando se intenta acceder una funcionalidad de administrador siendo un usuario estándar se bloquea el acceso. \\
\textbf{Precondición} & Siendo un usuario estándar, intentar acceder a un conjunto de datos el cuál es solo accesible para los administradores. \\
\rowcolor{gray!20} \textbf{Acciones} & Comprobar el rol del usuario. \\
\textbf{Postcondición} & Se notifica al usuario y se le redirige a la ventana de errores. \\
\hline
\end{longtable}

%CP8
\begin{longtable}{>{\hspace{0pt}}m{0.25\linewidth}>{\hspace{0pt}}m{0.764\linewidth}}
\label{CP8}
\caption{CP-8 Añadir traducciones}\\ 
\hline
\rowcolor{gray!20} \multicolumn{1}{|>{\hspace{0pt}}m{0.25\linewidth}|}{\textbf{CP-8}} & \multicolumn{1}{>{\hspace{0pt}}m{0.764\linewidth}|}{Añadir traducciones} \endfirsthead 
\hline
\textbf{Importancia} & Media \\
\rowcolor{gray!20} \textbf{Descripción} & Cuando se añade una traducción nueva debe estar disponible para todos los usuarios. \\
\textbf{Precondición} & Añadir una nueva traducción a una palabra. \\
\rowcolor{gray!20} \textbf{Acciones} & Añadir esa tupla en la base de datos para que los conjuntos sen traducidos. \\
\textbf{Postcondición} & Los usuarios dejan de ver la palabra sin traducir y ahora ven la traducción. \\
\hline
\end{longtable}
