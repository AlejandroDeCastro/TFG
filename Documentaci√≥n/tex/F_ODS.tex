\apendice{Anexo de sostenibilización curricular}

\begin{comment}
Este anexo incluirá una reflexión personal del alumnado sobre los aspectos de la sostenibilidad que se abordan en el trabajo.
Se pueden incluir tantas subsecciones como sean necesarias con la intención de explicar las competencias de sostenibilidad adquiridas durante el alumnado y aplicadas al Trabajo de Fin de Grado.

Más información en el documento de la CRUE \url{https://www.crue.org/wp-content/uploads/2020/02/Directrices_Sosteniblidad_Crue2012.pdf}.

Este anexo tendrá una extensión comprendida entre 600 y 800 palabras.
\end{comment}

\section{Introducción}
En este anexo se incluye una reflexión personal sobre los aspectos de la sostenibilidad que se abordan en el trabajo.

La sostenibilidad es un aspecto fundamental en el desarrollo de aplicaciones tecnológicas, especialmente aquellas que manejan datos abiertos. Este capítulo describe cómo la aplicación desarrollada durante este proyecto contribuye a la sostenibilidad ambiental, social y económica.

\section{Sostenibilidad Ambiental}
\subsection{Impacto Positivo}
La aplicación facilita el acceso a datos sobre la calidad del aire, el consumo de energía, y otros indicadores ambientales que estén disponibles. Esta información permite a las autoridades y a los ciudadanos tomar decisiones informadas que promuevan un entorno urbano más saludable.

\subsection{Gestión de Recursos}
Al proporcionar datos abiertos, la aplicación ayuda a mejorar la planificación urbana y la eficiencia en el uso de recursos, contribuyendo así a una gestión más sostenible. Como restricciones de tráfico en determinados distritos, reforzar líneas de transporte cuyos aforos se aproximen al máximo o incluso crear nuevas líneas de transporte público en aquellas zonas donde sea necesario.

\section{Sostenibilidad Social}
\subsection{Transparencia y Participación}
El acceso libre a datos públicos fomenta la transparencia gubernamental y la participación ciudadana, empoderando a los individuos para involucrarse en la toma de decisiones que afectan a su comunidad.

\subsection{Inclusión}
La aplicación promueve la equidad al proporcionar acceso igualitario a información relevante, reduciendo así las desigualdades sociales.

\section{Sostenibilidad Económica}
\subsection{Eficiencia y Ahorro}
El uso de datos abiertos permite optimizar los recursos económicos de la ciudad, mejorando la eficiencia en diferentes sectores y fomentando un desarrollo económico sostenible.

\section{Educación y Conciencia}
\subsection{Potencial Educativo}
La aplicación actúa como una herramienta educativa, ofreciendo datos que pueden ser utilizados en investigaciones académicas y estudios.

\subsection{Sensibilización}
La disponibilidad de datos abiertos aumenta la conciencia sobre problemas ambientales y sociales, incentivando a la ciudadanía a adoptar prácticas más sostenibles.

\section{Innovación y Tecnología}
\subsection{Uso de Tecnologías Sostenibles}
El desarrollo de la aplicación incluye la implementación de tecnologías eficientes y el uso de servidores de bajo consumo energético, asegurando un impacto ambiental reducido.

\section{Conclusiones}
El proyecto aborda competencias esenciales como la comprensión del impacto de la actividad profesional en el medio ambiente y la sociedad, el trabajo en equipos multidisciplinares, y la aplicación de un enfoque holístico para resolver problemas socioambientales.

Esto se puede ver reflejado en cada uno de los distintos conjuntos de datos que se pueden incluir en la aplicación, ya que, todos ellos contribuyen a la sostenibilidad, ya sea económica, social o ambiental. 

