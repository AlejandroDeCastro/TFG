\capitulo{4}{Técnicas y herramientas}

\begin{comment}
Esta parte de la memoria tiene como objetivo presentar las técnicas metodológicas y las herramientas de desarrollo que se han utilizado para llevar a cabo el proyecto. Si se han estudiado diferentes alternativas de metodologías, herramientas, bibliotecas se puede hacer un resumen de los aspectos más destacados de cada alternativa, incluyendo comparativas entre las distintas opciones y una justificación de las elecciones realizadas. 
No se pretende que este apartado se convierta en un capítulo de un libro dedicado a cada una de las alternativas, sino comentar los aspectos más destacados de cada opción, con un repaso somero a los fundamentos esenciales y referencias bibliográficas para que el lector pueda ampliar su conocimiento sobre el tema.
\end{comment}


Postman es una herramienta utilizada para pobar y gestionar APIs, en este caso, para hacer consultas y escrituras en el servidor de Orion Context Broker. Ofrece una interfaz amigable que permite realizar peticiones HTTP y ver las repuestas del servidor.


XAMPP es una distribución gratuita y de código abierto que facilita la instalación y el uso de un entorno de desarrollo web en un sistema local. Su nombre es un acrónimo que representa los componentes principales que incluye: 

\begin{itemize}
    \item \textbf{X} (Cross-platform): Indica que es multiplataforma, compatible con Windows, Linux, y macOS \cite{xampp}.
    \item \textbf{A} (Apache): Incluye el servidor web Apache, que es uno de los servidores web más populares del mundo.
    \item \textbf{M} (MariaDB/MySQL): Proporciona el sistema de gestión de bases de datos MariaDB (anteriormente MySQL).
    \item \textbf{P} (PHP): Incluye el lenguaje de scripting PHP, que es ampliamente utilizado para el desarrollo web del lado del servidor.
    \item \textbf{P} (Perl): También incluye Perl, otro lenguaje de programación.
\end{itemize}

XAMPP simplifica la instalación y configuración de estos componentes, lo que lo hace ideal para desarrolladores que desean crear, probar y depurar aplicaciones web localmente antes de desplegarlas en un servidor en producción \cite{xamppdocs}.

MySQL es un sistema de gestión de bases de datos relacional (RDBMS) que utiliza el lenguaje SQL (Structured Query Language) para gestionar y manipular datos. Es uno de los sistemas de bases de datos más utilizados en el mundo, especialmente en aplicaciones web \cite{mysql}. Algunas características clave de MySQL incluyen:

\begin{itemize}
    \item \textbf{Alto Rendimiento}: Optimizado para rendimiento y escalabilidad, capaz de manejar grandes cantidades de datos y transacciones.
    \item \textbf{Confiabilidad}: Proporciona características avanzadas como recuperación ante fallos y replicación para garantizar la integridad y disponibilidad de los datos.
    \item \textbf{Flexibilidad}: Soporta múltiples motores de almacenamiento (InnoDB, MyISAM, etc.) que permiten elegir el más adecuado según las necesidades específicas.
    \item \textbf{Compatibilidad}: Es compatible con múltiples plataformas y puede integrarse con una amplia variedad de lenguajes de programación y herramientas de desarrollo.
\end{itemize}

MySQL es utilizado por muchas aplicaciones y sitios web de alto tráfico, como Facebook, Twitter, y YouTube, debido a su eficiencia y robustez. Es una elección común para desarrolladores que buscan una solución de base de datos relacional confiable y de alto rendimiento \cite{mysqlcookbook}.


