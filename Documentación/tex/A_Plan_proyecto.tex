\apendice{Plan de Proyecto Software}

\section{Introducción}

En este apartado se procede a explicar la planificación temporal del proyecto, detallando cada uno de los Sprints desde el comienzo del proyecto hasta el final del mismo. También se aborda un estudio de viabilidad del proyecto, tanto económica como legal.

\section{Planificación temporal}

Como ya se ha explicado en la sección de técnica metodológica, la distribución temporal del proyecto se ha realizado mediante \textit{Sprints} de aproximadamente dos semanas de duración.
Durante los primeros seis \textit{Sprints} se ha usado la herramienta de gestión de tareas \textit{Microsoft Planner}

\subsection{Sprint 1: Primeros pasos (23/11/2023 - 20/12/2023)}
Durante la primera reunión se barajaron diferentes propuestas de proyecto, pero hubo una que me despertó especial interés y me acabé decantando por ella. \textit{Open Data in Smart Cities}.
Se establecieron las primeras tareas, como investigar acerca de \textit{FIWARE}, descargar la plantilla de memoria, pensar en las diferentes herramientas para la organización del proyecto (Repositorio, planificador de tareas, control de versiones...). 

Durante el Sprint, se terminaron por decidir dichas herramientas, se creó el repositorio y se incorporaron en el planificador de tareas las primeras tareas para empezar con la investigación y profundización en el ámbito del proyecto.
También se comenzó con la investigación de \textit{Flask}, el \textit{framework} usado para el desarrollo de la aplicación en \textit{Python}. Como finalmente se aplazó la reunión programada para esas dos semanas, se comenzó con una pequeña demo de tratamiento de datos con \textit{Flask}.

\imagen{Sprint 1}{Sprint 1 - Tareas}

\subsection{Sprint 2: Investigación \textit{FIWARE} y desarrollo de la aplicación (20/12/2023 - 17/01/2024)}

Durante este Sprint se continuó con la investigación acerca de \textit{FIWARE}, y se trabajó en un menú para el usuario y el tratamiento de datos.
También se comenzó con una demo paralela con interfaces gráficas, para decidir si finalmente se usaba \textit{Flask} o se desarrollaba la aplicación con interfaces gráficas.

\imagen{Sprint 2}{Sprint 2 - Tareas}

\subsection{Sprint 3: Despliegue del servidor OCB (17/01/2024 - 31/01/2024)}

En esta etapa se investigó acerca del funcionamiento de \textit{Docker} y se comenzó con el despliegue del \textit{Orion Context Broker}.
También se creó un catalogo de datos anexado en la propia tarea del \textit{Planner}, el cual contiene los diferentes enlaces y los formatos que contenían. Aquí se empezó a replantear el uso de del modelo de información \textit{NGSI-LD}, debido a la falta de conjuntos con ese modelo.
También se desarrollo un \textit{Script} que transformaba ficheros en formato \textit{JSON} al modelo \textit{NGSI-LD}, dicho \textit{script} sería descartado posteriormente por su escasa eficacia y funcionalidad.

\imagen{Sprint 3}{Sprint 3 - Tareas}

\subsection{Sprint 4: Despliegue total y testeo del \textit{OCB} (31/01/2024 - 14/02/2024)}

Durante esta etapa se siguió con el desarrollo del servidor del \textit{Orion Context Broker} con el apoyo de la herramienta \textit{Postman}, con la cuál se podía ir comprobando que la base de datos está guardando bien las entidades y es capaz de devolverlas.

\imagen{Sprint 4}{Sprint 4 - Tareas}

\subsection{Sprint 5: Simulador (14/02/2024 - 14/03/2024)}

En esta etapa del proyecto afrontamos el problema de la falta de datos con el modelo \textit{NGSI-LD}, desarrollando un \textit{script} que simula un conjunto de datos con este modelo.
Este simulador representaba un conjunto de \textit{parkings}, cuyos sensores publican cada 30 segundos el estado actual de las entidades (\textit{parkings}) en la base de datos, para ser consultadas desde la aplicación.

Al no tener una caché el \textit{OCB}, no se podía guardar el estado anterior de las entidades, por lo que no se tenía un histórico de las mismas, esta funcionalidad aportaría grandes funcionalidades como análisis de patrones o predecir el estado de las entidades. Por lo tanto, durante esta etapa se investigó acerca de alguna opción que permitiese guardar históricos de las entidades.

Existe una herramienta que forma parte de la plataforma \textit{FIWARE}, llamada \textit{Cygnus}, la cual permite guardar las actualizaciones de las entidades. Esta herramienta se intentó implementar para este caso y por complejidad y poca funcionalidad (ya que es solo un simulador y se puede ir guardando un histórico antes de publicar los datos en el servidor, agilizando así la aplicación), no se terminó de implantar. Pero puede ser útil para futuras versiones.

También durante Sprint se desarrolló una demo de una plantilla de \textit{PowerBi} para dichos datos, para intentar integrarla en aplicación. El inconveniente que se encontró es que \textit{PowerBi} es un software de pago, por lo tanto, no se pudo finalmente integrar en la aplicación.


\imagen{Sprint 5}{Sprint 5 - Tareas}

\subsection{Sprint 6: Nuevas funcionalidades en la aplicación (14/03/2024 - 24/04/2024)}

Durante esta etapa la aplicación ya estaba cogiendo forma, por lo que se comenzó con los cambios estéticos, para ofrecer al usuario una mayor simpleza y pueda tener una experiencia más amigable con la aplicación.
También se añadieron algunas funcionalidades como la de los registros en la aplicación, o alguna plantilla más visual con la librería \textit{Dash}.

También se investigó e incorporó la manera de autentificar a los usuarios con un \textit{login}.

\imagen{Sprint 6}{Sprint 6 - Tareas}

\subsection{Sprint 7: (24/04/2024 - 15/05/2024)}

A partir de aquí se llevó a cabo un cambio de herramienta para la gestión de tareas. Hasta entonces se había estado usando \textit{Microsoft Planner}, pero surgió una alternativa, la cual es mucho más práctica porque permitía enlazar las tareas con los \textit{commits} y tener un mejor control de las versiones y de las tareas. Esta alternativa es ofrecida por \textit{GitHub}, nuestro repositorio y herramienta de control de versiones. Mediantes los \textit{issues} incorporados en dicha herramienta se podía lograr una mejor integración de las herramientas y una simplificación del trabajo muy destacable.

En esta etapa

\subsection{Sprint 8 15/05/2024 - 31/05/2024}

\subsection{Sprint 9: Organización de la aplicación 31/05/2024 - 14/06/2024}

\subsection{Sprint 10: Recta final 14/06/2024 - 25/05/2024}

\subsection{Sprint 11: Últimos detalles y anexos (25/05/2024 - 03/05/2024)}



\section{Estudio de viabilidad}

\subsection{Viabilidad económica}

\subsection{Viabilidad legal}


