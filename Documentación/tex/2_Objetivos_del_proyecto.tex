\capitulo{2}{Objetivos del proyecto}
\begin{comment}
Este apartado explica de forma precisa y concisa cuales son los objetivos que se persiguen con la realización del proyecto. Se puede distinguir entre los objetivos marcados por los requisitos del software a construir y los objetivos de carácter técnico que plantea a la hora de llevar a la práctica el proyecto.
\end{comment}

El objetivo principal del proyecto es el desarrollo de una aplicación de datos abiertos en \textit{Smart Cities} utilizando la plataforma FIWARE \cite{fiware}. 

\section{Objetivos marcados por los requisitos del \textit{software}}\label{objetivos-generales}

\begin{itemize}
\tightlist
\item
  Permitir a cualquier usuario el acceso a un conjunto de datos de una ciudad de manera rápida y precisa.
\item
  Ofrecer al usuario una experiencia sencilla y cómoda, enfocada a cualquier tipo de público. También se busca que esta experiencia sea personalizada, pudiendo personalizar cada conjunto por cada usuario.
\item
  Facilitar al usuario la incorporación de nuevos conjuntos de datos, disponibles para todos los usuarios.
\item
  Permitir al usuario guardar un registro de los conjuntos de datos que desee para un posterior análisis.
\end{itemize}



\section{Objetivos de carácter técnico}\label{objetivos-personales}

\begin{itemize}
\tightlist

\item Investigar sobre la implantación de \textit{FIWARE} y ponerlo en marcha.
\item Aplicar y ampliar los conocimientos adquiridos durante el grado.
\item Aprender sobre el desarrollo de aplicaciones web.
\item Ampliar y consolidar conocimientos sobre análisis y tratamiento de datos.
\item Usar GitHub para el control de versiones.
\item Implementar una base de datos en la aplicación para guardar los diferentes datos necesarios.
\item Ampliar los conocimientos de la herramienta de desarrollo de documentos \LaTeX{}.
\item Usar una metodología ágil \textit{Scrum} para el desarrollo del proyecto.
\item Aplicar y ampliar los conocimientos sobre \textit{Python}, desarrollo web, bases de datos, gestión de proyectos, estructuras de datos y análisis de \textit{software} adquiridos en el grado.


\end{itemize}