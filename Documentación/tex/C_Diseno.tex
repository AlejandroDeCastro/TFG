\apendice{Especificación de diseño}

\section{Introducción}
En este anexo se define la resolución de los objetivos y especificaciones de los anexos anteriores.

Su estructura es la siguiente:
\begin{itemize}
    \item \textbf{Diseño de datos:} En este aparatado se detalla el modelado de datos de la aplicación mediante el diagrama Entidad-Relación\ref{e-r}, el modelo relacional\ref{modelo-relacional} y el diccionario de datos\ref{diccionario-de-datos} de la aplicación.
    \item \textbf{Diseño procedimental:}
    \item \textbf{Diseño arquitectónico:}
\end{itemize}

\section{Diseño de datos}
\subsection{Diagrama Entidad-Relación}\label{e-r}

Un diagrama entidad-relación (ERD) describe los datos o información de un sistema mediante modelado de sus entidades, atributos y las relaciones entre esas entidades. \cite{Chen1976}

\subsection{Modelo Relacional}\label{modelo-relacional}

Un modelo relacional es una forma de estructurar y consultar datos en una base de datos. En este modelo, los datos se organizan en tablas (o relaciones), donde cada tabla está compuesta por filas (tuplas) y columnas (atributos). \cite{Codd1970}

\subsection{Diccionario de datos}\label{diccionario-de-datos}

Un diccionario de datos es una herramienta crucial en la gestión de bases de datos que almacena definiciones y descripciones de los elementos de datos utilizados en el sistema. Proporciona una visión centralizada y coherente de los datos, incluyendo detalles sobre los nombres, tipos, formatos, y relaciones entre los elementos de datos. \cite{Kent1983}




\section{Diseño procedimental}

\section{Diseño arquitectónico}


