\capitulo{7}{Conclusiones y Líneas de trabajo futuras}
\begin{comment}
Todo proyecto debe incluir las conclusiones que se derivan de su desarrollo. Éstas pueden ser de diferente índole, dependiendo de la tipología del proyecto, pero normalmente van a estar presentes un conjunto de conclusiones relacionadas con los resultados del proyecto y un conjunto de conclusiones técnicas. 
Además, resulta muy útil realizar un informe crítico indicando cómo se puede mejorar el proyecto, o cómo se puede continuar trabajando en la línea del proyecto realizado. 
\end{comment}

\section{Conclusiones}

Tras finalizar el proyecto se han obtenido diferentes conclusiones, las cuales he clasificado en tres grupos.
\begin{itemize}
    \item \textbf{Conclusiones del proyecto}
    \item \textbf{Conclusiones técnicas}
    \item \textbf{Conclusiones personales}
\end{itemize}

\subsection{Conclusiones del proyecto}
El proyecto ha concluido con éxito, cumpliendo todos y cada uno de los objetivos marcados al comienzo del mismo. Se ha conseguido realizar una aplicación funcional, cubriendo las necesidades de los usuarios y un costo de recursos mínimo.

\subsection{Conclusiones técnicas}
\begin{itemize}
    \item La aplicación es robusta, tiene un tratamiento de excepciones preparado para cubrir todos los errores. 
    \item Es escalable y fácilmente mantenible. Permite añadir nuevos formatos de manera rápida y simple.
    \item Tiene una interfaz amigable con el usuario, permitiendo que pueda personalizar los conjuntos que desea consultar y añadir nuevos. 
    \item Posee cifrado para que las contraseñas estén siempre seguras.
    \item Su implementación y uso es simple.
\end{itemize}

\subsection{Conclusiones personales}


\section{Posibles mejoras futuras}