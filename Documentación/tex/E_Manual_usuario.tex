\apendice{Documentación de usuario}

\section{Introducción}
En este anexo


\section{Requisitos de usuarios}

Navegador

\section{Instalación}

La aplicación al ser desplegada carece de 
El usuario no tiene que descargar

\section{Manual del usuario}


\subsection{Iniciar sesión}\label{inicio-sesión}
Cuando el usuario entra en el enlace de la aplicación, lo primero que se le muestra es la pantalla de inicio de sesión donde debe introducir su usuario y contraseña.
IMAGEN DE LOGIN

En caso de introducir mal la contraseña se le notificará al usuario con un mensaje de error.
IMAGEN CONTRASEÑA MAL

Y en caso de no existir un usuario con ese nombre en la base de datos se le notificará también con un error.
IMAGEN USUARIO NO EXISTE


\subsection{Crear una cuenta}\label{crear-cuenta}

Si el usuario no tiene una cuenta en la aplicación puede crear una pulsando en el enlace que dice "CREAR CUENTA", será redirigido a una pantalla IMAGEN REGISTRO donde puede crear una cuenta introduciendo un nombre de usuario, nombre completo y la contraseña dos veces.

IMAGEN REGISTRO

En caso de existir un usuario con ese nombre de usuario se le notificará.

IMAGEN USUARIO REPETIDO

Y en el caso de no coincidir las dos contraseñas también se le notificará al usuario.

IMAGEN CONTRASEÑA NO COINCIDEN


\subsection{Pantalla principal}\label{home}

Una vez introducido los datos correctamente, el usuario solo tiene que pulsar el botón "CREAR CUENTA" y la cuenta será creada. Posteriormente será redirigido la pantalla de login IMAGEN LOGIN donde deberá introducir sus datos para continuar.

Una vez iniciada sesión en la aplicación el usuario será redirigido a la pantalla principal, la cual varía según el rol del usuario. Si se acaba de crear la cuenta por defecto será un usuario estándar y la pantalla que verá será la siguiente:
IMAGEN HOME

Por el contrario, si es un usuario con permiso de administrador, esta será la pantalla principal para ese usuario:
IMAGEN HOME ADMIN

En esta pantalla principal ambos usuarios pueden ver en la parte inferior izquierda los conjuntos que tienen marcados en favoritos y acceder a ellos pulsando sobre ellos o seleccionando la ciudad y el conjunto en el cuadrado inferior de la parte derecha de la pantalla. Ambas opciones redirigirán al usuario a la pantalla de visualizar conjunto \ref{visualizar-conjunto}.

Los usuarios estándar podrán consultar dentro del recuadro verde los registros que tienen en ese momento guardando.
IMAGEN AMPLIADA REGISTROS.

Y los usuarios con rol de administrador tendrán en su defecto un menú con tres opciones que redirigirán al usuario a la interfaz de gestión de usuarios \ref{gestión-usuarios}, la lista de conjuntos disponibles \ref{consultar-conjuntos} o la pantalla de gestión de traducciones \ref{gestión-traducciones}.


Ambos tipos de usuarios, en la barra de navegación, podrán consultar conjuntos disponibles \ref{consultar-conjuntos} pulsando en el botón "CONJUNTOS", consultar los registros que tienen actualmente guardando, pulsando en el botón "REGISTROS" \ref{consultar-registros}, consultar la pantalla de ayuda \ref{consulta-ayuda} o cerrar la sesión pulsando el botón en rojo "CERRAR SESIÓN". Si el usuario selecciona esta opción será redirigido a la pantalla de inicio de sesión \ref{inicio-sesión}

\subsection{Consultar los conjuntos de datos disponibles}\label{consultar-conjuntos}

IMAGEN CONJUNTOS

Desde esta pantalla el usuario puede consultar una lista con los conjuntos disponibles en la aplicación y acceder a ellos pulsando en el botón consultar azul de cada conjunto. Será direccionado al la visualización del conjunto seleccionado \ref{visualizar-conjunto}.

También el usuario puede ordenar de manera ascendente o descendente los conjuntos en función de la ciudad, el tipo de conjunto...MAS, pulsando en el nombre del campo en el cabecero de la tabla.

Si desea añadir un conjunto el usuario solo tiene que rellenar los campos de la parte superior de  la pantalla y posteriormente pulsar el botón "AÑADIR", y automáticamente se añadirá el conjunto a la lista.

Los administradores tendrán un botón rojo "Eliminar" en cada conjunto. Esto permite a los adminitradores elminar el conjunto que seleccionen.
IMAGEN CONJUNTOS ADMIN

\subsection{Consultar registros}\label{consultar-registros}

\subsection{Visualizar conjunto en tiempo real}\label{visualizar-conjunto}

\subsection{Gestionar traducciones}\label{gestión-traducciones}

\subsection{Gestionar usuarios}\label{gestión-usuarios}

\subsection{Ayuda}\label{consulta-ayuda}

\subsection{Errores}\label{errores}
