\capitulo{3}{Conceptos teóricos}

\section{FIWARE}
FIWARE es una plataforma impulsada por la Unión Europea para el desarrollo y despliegue 
global de aplicaciones de Internet del Futuro. Proporciona APIs abiertas y componentes para 
gestionar información de contexto en soluciones inteligentes, gemelos digitales y espacios de 
datos \cite{fiware2024}. En resumen, FIWARE es una base tecnológica que permite crear soluciones 
innovadoras en áreas como Smart Cities, Agricultura Inteligente, Energía Inteligente, Industria 
Inteligente y Gestión del Agua. 

\section{NGSI-LD}
NGSI-LD es un modelo de información y API para editar, consultar y suscribirse a información 
de contexto. Está destinado a facilitar el intercambio abierto y la compartición de información 
estructurada entre diferentes partes interesadas. Se utiliza en diversos ámbitos de aplicación, 
como Ciudades Inteligentes, Industria Inteligente, Agricultura Inteligente, y más generalmente 
para el Internet de las Cosas, Sistemas Ciberfísicos y Gemelos Digitales \cite{etsi2024}.

\section{Orion Context Broker}
El principal y único componente obligatorio de cualquier plataforma o solución desarrollada con FIWARE es el Orion Context Broker (OCB), el cual aporta una función fundamental en cualquier solución inteligente: administrar la información de contexto, consultarla y actualizarla \cite{orion2024}. El OCB permite la publicación de información de contexto por entidades (llamados productores de contexto) por ejemplo los sensores, de manera que la información de contexto publicada se encuentre disponible para otras entidades (consumidores de contexto) los cuales están interesados en procesar la información, por ejemplo una aplicación para smartphones que usa la información de los sensores.




En aquellos proyectos que necesiten para su comprensión y desarrollo de unos conceptos teóricos de una determinada materia o de un determinado dominio de conocimiento, debe existir un apartado que sintetice dichos conceptos.

Algunos conceptos teóricos de \LaTeX{} \footnote{Créditos a los proyectos de Álvaro López Cantero: Configurador de Presupuestos y Roberto Izquierdo Amo: PLQuiz}.

\section{Secciones}

Las secciones se incluyen con el comando section.

\subsection{Subsecciones}

Además de secciones tenemos subsecciones.

\subsubsection{Subsubsecciones}

Y subsecciones. 


\section{Referencias}

Las referencias se incluyen en el texto usando cite~\cite{wiki:latex}. Para citar webs, artículos o libros~\cite{koza92}, si se desean citar más de uno en el mismo lugar~\cite{bortolot2005, koza92}.


\section{Imágenes}

Se pueden incluir imágenes con los comandos standard de \LaTeX, pero esta plantilla dispone de comandos propios como por ejemplo el siguiente:

\imagen{escudoInfor}{Autómata para una expresión vacía}{.5}



\section{Listas de items}

Existen tres posibilidades:

\begin{itemize}
	\item primer item.
	\item segundo item.
\end{itemize}

\begin{enumerate}
	\item primer item.
	\item segundo item.
\end{enumerate}

\begin{description}
	\item[Primer item] más información sobre el primer item.
	\item[Segundo item] más información sobre el segundo item.
\end{description}
	
\begin{itemize}
\item 
\end{itemize}

\section{Tablas}

Igualmente se pueden usar los comandos específicos de \LaTeX o bien usar alguno de los comandos de la plantilla.

\tablaSmall{Herramientas y tecnologías utilizadas en cada parte del proyecto}{l c c c c}{herramientasportipodeuso}
{ \multicolumn{1}{l}{Herramientas} & App AngularJS & API REST & BD & Memoria \\}{ 
HTML5 & X & & &\\
CSS3 & X & & &\\
BOOTSTRAP & X & & &\\
JavaScript & X & & &\\
AngularJS & X & & &\\
Bower & X & & &\\
PHP & & X & &\\
Karma + Jasmine & X & & &\\
Slim framework & & X & &\\
Idiorm & & X & &\\
Composer & & X & &\\
JSON & X & X & &\\
PhpStorm & X & X & &\\
MySQL & & & X &\\
PhpMyAdmin & & & X &\\
Git + BitBucket & X & X & X & X\\
Mik\TeX{} & & & & X\\
\TeX{}Maker & & & & X\\
Astah & & & & X\\
Balsamiq Mockups & X & & &\\
VersionOne & X & X & X & X\\
} 
